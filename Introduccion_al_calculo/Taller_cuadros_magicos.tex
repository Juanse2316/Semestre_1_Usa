\documentclass[12pt]{article}

% --- Codificación y español ---
\usepackage[utf8]{inputenc}
\usepackage[T1]{fontenc}
\usepackage[spanish]{babel}

% --- Márgenes y utilidades ---
\usepackage[margin=2.2cm]{geometry}
\usepackage{amsmath,amssymb}
\usepackage{array}
\usepackage{enumitem}

% --- TikZ y colores ---
\usepackage{tikz}
\usepackage[dvipsnames]{xcolor} % ForestGreen disponible

% ===== Encabezado =====
\newcommand{\Nombre}{Juan Ladino}
\newcommand{\Materia}{\rule{5cm}{0.4pt}} % línea para escribir la materia

% ===== Estilo de los números dados (verdes) =====
\tikzset{
  msnum/.style={ font=\bfseries}%
  % Si prefieres sin color, usa:
  % msnum/.style={font=\bfseries}
}

% ===== Macro para 3x3 (r1c1 ... r3c3) =====
\newcommand{\magicsquare}[9]{%
\begin{tikzpicture}[scale=1]
  \draw[very thick] (0,0) rectangle (3,3);
  \draw[thick] (1,0)--(1,3) (2,0)--(2,3) (0,1)--(3,1) (0,2)--(3,2);
  \node[msnum] at (0.5,2.5) {#1};
  \node[msnum] at (1.5,2.5) {#2};
  \node[msnum] at (2.5,2.5) {#3};
  \node[msnum] at (0.5,1.5) {#4};
  \node[msnum] at (1.5,1.5) {#5};
  \node[msnum] at (2.5,1.5) {#6};
  \node[msnum] at (0.5,0.5) {#7};
  \node[msnum] at (1.5,0.5) {#8};
  \node[msnum] at (2.5,0.5) {#9};
\end{tikzpicture}%
}

% ===== Macro para rotular cada cuadro =====
\newcommand{\cuadro}[2]{%
  \begin{tabular}{@{}c@{}}
    #1\\[-0.4em]
    \footnotesize\textit{Cuadro #2}
  \end{tabular}%
}

\begin{document}
\noindent\textbf{Nombre:} \Nombre \hfill
\textbf{Materia:} \Materia \hfill
\textbf{Fecha:} \rule{3cm}{0.4pt}
\par\medskip\hrule\medskip

\section{Cuadros Mágicos}

\begin{center}
\setlength{\tabcolsep}{1.4cm}

% ================= PRIMERA TABLA =================
\begin{tabular}{cc}
% 1) Cuadro 1
\magicsquare{5}{6}{-8}{-12}{1}{14}{10}{-4}{-3} &
% 2) Cuadro 2
\magicsquare{-8}{10}{-11}{-6}{ -3 }{0}{5}{-16}{2} \\[2.0em]
% 3) Cuadro 3
\magicsquare{7}{0}{-1}{-6}{2}{10}{5}{4}{-3} &
% 4) Cuadro 4
\magicsquare{-2}{-7}{-6}{-9}{-5}{-1}{-4}{-3}{-8}\\[2.0em]
% 5
\magicsquare{0.3}{1.0}{0.5}{0.8}{0.6}{0.4}{0.7}{0.2}{0.9} &
% 6
\magicsquare{1/12}{3/4}{1/6}{5/12}{1/3}{1/4}{1/2}{\(\frac{-1}{12}\)}{7/12}
\\[2.0em]
\end{tabular}

\vspace{2em} % espacio entre tablas

% ================= SEGUNDA TABLA =================
\begin{tabular}{cc}
% 7
\magicsquare{11.8}{35.8}{31}{45.4}{26.2}{7}{21.4}{16.6}{40.6} &
% 8
\magicsquare{2}{7}{6}{9}{5}{1}{4}{3}{8}
\\[2.0em]
% 9
\magicsquare{8}{1}{6}{3}{5}{7}{4}{9}{2} &
% 10
\magicsquare{-3}{4}{-1}{2}{0}{-2}{1}{-4}{3}
\\[2.0em]
% 11
\magicsquare{-3}{2}{-5}{-4}{-2}{0}{1}{-6}{-1} &
% 12
\magicsquare{7/10}{-3/5}{\(\frac{11}{10}\)}{4/5}{2/5}{0}{\(\frac{-3}{10}\)}{7/5}{1/10}\\[2.0em]
% 13
\magicsquare{5.3}{13.3}{11.7}{16.5}{10.1}{3.7}{8.5}{6.9}{14.9} &
\end{tabular}
\end{center}

\section{Estrategias de solución}
\begin{itemize}
  \item \textbf{Encuentra la suma mágica con el centro.}\\
        Multiplica el número del centro por \(3\). Ese total (\(S\)) es el que deben sumar todas las filas, columnas y diagonales.
  \item \textbf{Opuestos alrededor del centro.}\\
        Cualquier par de casillas opuestas (a través del centro) suma \(2\times c\), donde \(c\) es el centro. Si conoces una casilla \(a\), la opuesta vale \(2c - a\).
  \item \textbf{Dos centros vecinos \(\to\) esquina opuesta.}\\
        Suma el centro de arriba y el centro de la izquierda y divide entre \(2\): obtienes la esquina opuesta a ambos.
  \item \textbf{Completar líneas con dos números.}\\
        Si ya conoces \(S\) y tienes dos números \(a\) y \(b\) en una misma fila/columna/diagonal, el tercero es \(S - (a+b)\).
  \item \textbf{Primero la “cruz”.}\\
        Empieza por la fila y la columna que pasan por el centro; suelen tener más pistas.
  \item \textbf{Dos esquinas \(\to\) centro.}\\
        Si conoces dos esquinas opuestas \(a\) y \(b\), el centro es su promedio: \(c=\frac{a+b}{2}\).
\end{itemize}

\section{Taller 1: Cuadros Mágicos 3x3}

\begin{center}
\begin{tabular}{cc}
% 1
\cuadro{\magicsquare{}{17}{}{11}{}{}{}{}{13}}{1} &
% 2 (resuelto)
\cuadro{\magicsquare{\(\frac{1}{24}\)}{\(\frac{5}{3}\)}{\(\frac{11}{12}\)}
                      {\(\frac{7}{4}\)}{\(\frac{7}{8}\)}{0}
                      {\(\frac{5}{6}\)}{\(\frac{1}{12}\)}{\(\frac{41}{24}\)}}{2} \\[1.6em]
% 3 (vacío mostrado; puedes sustituir por la versión llena si quieres)
\cuadro{\magicsquare{19}{17}{33}{37}{23}{9}{13}{29}{27}}{3} &

\cuadro{\magicsquare{1.25}{}{}{}{2.15}{}{}{}{0.75}}{4} \\[1.6em]
% 5
\cuadro{\magicsquare{}{31}{}{41}{}{}{}{}{47}}{5} &
% 6
\cuadro{\magicsquare{}{$\pi$}{}{$e$}{}{}{}{}{$\phi$}}{6}
\end{tabular}
\end{center}

\begin{center}
\begin{tabular}{cc}
% 7
\cuadro{\magicsquare{-7}{}{}{}{11}{}{}{}{5}}{7} &
% 8
\cuadro{\magicsquare{$0.\overline{3}$}{}{}{}{$1.\overline{6}$}{}{}{}{$2.\overline{7}$}}{8} \\[1.6em]
% 9
\cuadro{\magicsquare{}{-13}{}{-17}{}{}{}{}{-19}}{9} &
% 10
\cuadro{\magicsquare{}{17}{}{11}{}{}{}{}{13}}{10} \\[1.6em]
% 11
\cuadro{\magicsquare{\(\frac{1}{6}\)}{}{\(\frac{1}{3}\)}
                      {}{}{}
                      {\(\frac{2}{9}\)}{}{\(\frac{1}{18}\)}}{11} &
% 12
\cuadro{\magicsquare{}{31}{}{41}{}{}{}{}{47}}{12}
\end{tabular}
\end{center}

\section*{Explicaciones de cada cuadro}
\begin{enumerate}[label=\textbf{Cuadro \arabic*:}, leftmargin=*]

\item \textbf{No tiene solución.}
Opuestos fuerzan $c=m-4$, $g=m+4$ $\Rightarrow$ col.~1 $=3m+2\neq 3m$.

\item \textbf{Tiene solución.}

\item \textbf{Tiene solución.}

\item \textbf{No tiene solución.}
Opuestos: $a+i=2.00$ pero $2m=4.30$.

\item \textbf{No tiene solución.}
De los datos se obtiene col.~1 $=3m-22\neq 3m$.

\item \textbf{No tiene solución.}
Exigiría $e+\pi=2\phi$, lo cual es falso.

\item \textbf{No tiene solución.}
Centro $m=11\Rightarrow S=33$. Opuestos: $a+i=-7+5=-2\neq 22$.

\item \textbf{No tiene solución.}
Centro $m=\tfrac{5}{3}\Rightarrow 2m=\tfrac{10}{3}$; pero $a+i=\tfrac{28}{9}$.

\item \textbf{No tiene solución.}
De los datos se deduce col.~1 $=3m+8\neq 3m$.

\item \textbf{No tiene solución.}
Mismo patrón que el Cuadro 1.

\item \textbf{No tiene solución.}
De $a=\tfrac{1}{6}$, $i=\tfrac{1}{18}$ se fuerza $m=\tfrac{1}{9}$, pero $c+g=\tfrac{5}{9}\neq 2m=\tfrac{2}{9}$.

\item \textbf{No tiene solución.}
Mismo patrón que el Cuadro 5.

\end{enumerate}
\end{document}
