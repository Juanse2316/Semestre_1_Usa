\documentclass[12pt,letterpaper]{exam}

\usepackage[left=2cm,top=2cm,right=2cm,bottom=2.5cm]{geometry}
\usepackage{hyperref}
\usepackage[T1]{fontenc}
\usepackage[utf8]{inputenc}
\DeclareUnicodeCharacter{001C}{}
\usepackage[spanish,activeacute]{babel}
\decimalpoint

\usepackage{enumerate}
\usepackage{eurosym}
\usepackage{latexsym,amsmath,amsthm,amssymb,amsfonts,bbm,dsfont}
\usepackage{mathrsfs}

% --- TikZ ---
\usepackage{tikz}
\usetikzlibrary{calc,arrows.meta,intersections,positioning}
\tikzset{
  plane/.style={fill=gray!12, draw=black!50},
  line3d/.style={line width=0.9pt},
  pt/.style={circle,fill=black,inner sep=1.4pt},
  label/.style={font=\small}
}

\usepackage{graphicx}
\DeclareGraphicsExtensions{.pdf,.png,.jpg}
\usepackage{adjustbox}
\usepackage[dvipsnames]{xcolor}
\usepackage[font=small,labelfont=bf,textfont=it]{caption}
\usepackage{subcaption}
\usepackage{enumitem}
\usepackage{float}
\usepackage{multirow}
\usepackage{fancybox}

\graphicspath{{../assets/}{../Talleres_fundamentos/}{./}}

%--------------------------------------------------------------------
\newcommand{\base}[1]{\underline{\hspace{#1}}}
\newcommand{\uni}{Universidad Sergio Arboleda}
\newcommand{\fac}{\normalsize{Escuela de Ciencias Ex\'actas e Ingenier\'ia}}
\newcommand{\dep}{Matem\'aticas}
\newcommand{\mat}{Materia}
\newcommand{\tema}{}
\newcommand{\autor}{Profesor}
\newcommand{\espacio}[1]{\vspace{#1}}
\renewcommand{\arraystretch}{1.15}



%%%% AQUÍ DEFINES EL ENTORNO proposition %%%%
\theoremstyle{plain}
\newtheorem{proposition}{Proposición}
%%%%%%%%%%%%%%%%%%%%%%%%%%%%%%%%%%%%%%%%%%%%%

\pagestyle{headandfoot}
\footrule
\headrule
\firstpageheader{}{}{}
\firstpagefooter{}{\thepage $\,$ de \numpages}{}
\runningfooter{\uni}{\thepage $\,$ de \numpages}{}



\begin{document}
% ===== Encabezado institucional =====
\begin{tabular}{lr}
    \multirow{2}{*}{\includegraphics[height=1.2cm]{logosergio.png}} &
    {\textbf{\uni}} \\
    & {\textbf{\fac}} \\
    & {\textbf{\dep}} \\
    & {\textbf{\mat \tema}} \\
    & {\textit{\autor}} \\
    & {\textit{}}
\end{tabular}\\
\base{19.5cm}\\
% \textbf{Nombre}: \base{13.2cm} \quad \textit{Calificación}: \base{2cm} \\[6pt]
\textbf{Nombre}: \makebox[11.2cm]{\hfill Juan Sebastian Ladino Mendieta \hfill} \quad 
\textit{Calificaci\'on}: \base{2cm} \\[6pt]

\section{Taller (cuadrilateros)}

\subsection{Definiciones}
\begin{enumerate}
    \item Un cuadrilátero es convexo si y solo si sus diagonales se intersectan
    \item Un cuadrilátero es convexo si y solo si el interio de cada angulo del cuadrilátero
          contiene a un vertice del cuadrilátero.
    \item Un cuadrilátero no es convexo si y solo si se puede determinar un triángulo
          con los vertices del cuadrilátero de modo que el cuarto vertice está en el interior 
          del triángulo construido.
    \item Un cuadrilátero es convexo si y solo si cada recta que contien a cada lado del
          cuadrilátero deja a las otras dos vértices en el mismo semiplano.
\end{enumerate}

\subsection{Construción}
\begin{proposition}
Bajo los postulados de incidencia, de la regla, de separación del plano y los
teoremas de segmentación en semiplanos, existe al menos un cuadrilátero convexo usando la definición 4.
\end{proposition}

\begin{proof}
Por el postulado del plano y el de la mínima cantidad de puntos, existe un plano
$\alpha$ con al menos tres puntos no colineales. Sean $O,P,Q\in\alpha$ no colineales y
definamos las rectas
\[
  \ell = \overleftrightarrow{OP}, \qquad m = \overleftrightarrow{OQ}.
\]
Por el postulado de la recta, $\ell$ y $m$ son rectas bien definidas y se cortan en $O$.

Usando el postulado de la regla sobre $\ell$ y $m$, elegimos:
\[
 A,C \in \ell \setminus \{O\} \quad\text{con}\quad A-O-C,\qquad
 B,D \in m \setminus \{O\} \quad\text{con}\quad B-O-D.
\]
Consideramos ahora los cuatro segmentos
\[
  AB,\; BC,\; CD,\; DA
\]
y definimos el cuadrilátero $ABCD$ como la unión de dichos segmentos. Con una elección
genérica de $A,B,C,D$, los lados sólo se intersectan en sus vértices,
de modo que $ABCD$ es efectivamente un cuadrilátero.

Resta probar que $ABCD$ es convexo:
para cada lado, la recta que lo contiene deja a los otros dos vértices en el mismo
semiplano.

Sea, por ejemplo, el lado $AB$ y sea $r = \overleftrightarrow{AB}$ la recta que lo
contiene. Por el postulado de separación del plano, $r$ divide al plano $\alpha$ en
dos semiplanos. Queremos ver que los puntos $C$ y $D$ quedan en el mismo semiplano
determinado por $r$.

Primero probamos que $CD \cap r = \emptyset$.

Observemos que $A \in \ell \setminus m$ y $B \in m \setminus \ell$. Por tanto, la
recta $r = \overleftrightarrow{AB}$ no puede coincidir ni con $\ell$ ni con $m$.
Además, como $B \in r \cap m$, del postulado de la recta se sigue que
\[
r \cap m = \{B\},
\]
pues si hubiese otro punto $X \in r \cap m$ con $X \neq B$, las rectas $r$ y $m$
tendrían dos puntos comunes $B$ y $X$, y por el postulado de la recta serían la misma
recta, contradiciendo que $A \notin m$.

Ahora usamos el postulado de la regla sobre la recta $m$. Sea $f_m : m \to \mathbb{R}$
la biyección dada por dicho postulado, y fijamos la notación de modo que
\[
f_m(O) = 0,\qquad f_m(C) < 0 < f_m(D).
\]
Como hay infinitos puntos en $m$, podemos elegir $B$ de forma que
\[
f_m(B) < f_m(C) < 0 < f_m(D),
\]
manteniendo la condición $B-O-D$ (esto se cumple porque basta que $f_m(B) < 0 < f_m(D)$
para que $O$ esté entre $B$ y $D$).

Por definición de segmento, los puntos del segmento $CD$ son exactamente aquellos
$X \in m$ tales que
\[
f_m(C) \leq f_m(X) \leq f_m(D).
\]
Como $f_m(B) < f_m(C)$, se sigue que $B \notin CD$.

Supongamos ahora, para obtener una contradicción, que existe un punto
$X \in CD \cap r$. Entonces $X \in m$ (porque $C,D \in m$ y $CD \subset m$) y
$X \in r$, de modo que
\[
X \in r \cap m.
\]
Pero ya vimos que $r \cap m = \{B\}$, así que $X = B$. Esto contradice el hecho de que
$B \notin CD$. Por tanto,
\[
CD \cap r = \emptyset.
\]

Finalmente, aplicamos el teorema de \emph{segmento contenido en un semiplano} a la
recta $r$ y al segmento $CD$. Como $CD \cap r = \emptyset$, se concluye que los
extremos $C$ y $D$ están del mismo lado de $r$, es decir, pertenecen al mismo
semiplano determinado por la recta que contiene al lado $AB$.

De manera similar:
\begin{itemize}
  \item $\overleftrightarrow{BC}$ no corta al segmento $AD$, luego $A$ y $D$ quedan
        en el mismo semiplano.
  \item $\overleftrightarrow{CD}$ no corta al segmento $AB$, luego $A$ y $B$ quedan
        en el mismo semiplano.
  \item $\overleftrightarrow{DA}$ no corta al segmento $BC$, luego $B$ y $C$ quedan
        en el mismo semiplano.
\end{itemize}

Por tanto, $ABCD$ es un cuadrilátero convexo según la definición dada.
\end{proof}


\subsection{ángulos con cada vértice dentro}
\begin{proposition}[Caracterización angular de la convexidad]
Sea $ABCD$ un cuadrilátero. Consideremos las dos afirmaciones:

\begin{enumerate}
  \item[(i)] $ABCD$ es convexo si y sólo si cada recta que contiene a cada lado del
  cuadrilátero deja a los otros dos vértices en el mismo semiplano.
  
  \item[(ii)] $ABCD$ es convexo si y sólo si el interior de cada ángulo del
  cuadrilátero contiene a un vértice del cuadrilátero (el vértice opuesto).
\end{enumerate}

Entonces \textbf{(i) y (ii) son equivalentes}.
\end{proposition}

\begin{proof}
Recordemos la definición del interior de un ángulo:  
dado un ángulo $\angle XAY$ determinado por $\overrightarrow{AX}$ y
$\overrightarrow{AY}$, un punto $P$ está en el interior de $\angle XAY$ si y sólo si

\begin{itemize}
  \item $P$ está en el mismo semiplano que $X$ respecto de la recta $\overleftrightarrow{AY}$, y
  \item $P$ está en el mismo semiplano que $Y$ respecto de la recta $\overleftrightarrow{AX}$.
\end{itemize}

En particular, para el ángulo en $A$ del cuadrilátero $ABCD$ tomamos el ángulo
$\angle DAB$, determinado por $\overrightarrow{AD}$ y
$\overrightarrow{AB}$.

\medskip
\noindent\textbf{(i) $\Rightarrow$ (ii).}  
Supongamos que se cumple (i), es decir, que para cada lado la recta que lo contiene
deja a los otros dos vértices en el mismo semiplano.

Consideremos el ángulo en el vértice $A$, esto es, $\angle DAB$, y veamos que
su interior contiene al vértice opuesto $C$.

Por la condición (i) aplicada al lado $AB$, la $\overleftrightarrow{AB}$
deja a $C$ y $D$ en el mismo semiplano.  
Por la condición (i) aplicada al lado $AD$, la $\overleftrightarrow{AD}$
deja a $B$ y $C$ en el mismo semiplano.

De acuerdo con la definición de interior de $\angle DAB$, esto significa exactamente
que $C$ está en el interior de $\angle DAB$:

\begin{itemize}
  \item $C$ está en el mismo semiplano que $D$ respecto de $\overleftrightarrow{AB}$;
  \item $C$ está en el mismo semiplano que $B$ respecto de $\overleftrightarrow{AD}$.
\end{itemize}

Por tanto, el vértice opuesto $C$ pertenece al interior del ángulo en $A$.

Repitiendo el mismo razonamiento para los demás vértices:

\begin{itemize}
  \item en el ángulo en $B$ (ángulo $\angle ABC$) el vértice opuesto $D$
        pertenece a su interior;
  \item en el ángulo en $C$ (ángulo $\angle BCD$) el vértice opuesto $A$
        pertenece a su interior;
  \item en el ángulo en $D$ (ángulo $\angle CDA$) el vértice opuesto $B$
        pertenece a su interior.
\end{itemize}

Así se verifica la condición (ii): el interior de cada ángulo del cuadrilátero
contiene a un vértice del cuadrilátero (el opuesto).

\medskip
\noindent\textbf{(ii) $\Rightarrow$ (i).}  
Ahora supongamos que se cumple (ii), esto es, que el interior de cada ángulo del
cuadrilátero contiene al vértice opuesto.

Queremos probar que se cumple (i): para cada lado, la recta que lo contiene deja a
los otros dos vértices en el mismo semiplano.

Tomemos el lado $AB$ y veamos que la recta $r = \overleftrightarrow{AB}$ deja a
$C$ y $D$ en el mismo semiplano.

Por hipótesis, el vértice opuesto a $A$, que es $C$, está en el interior del
ángulo en $A$, es decir, en el interior de $\angle DAB$.  
Por la definición de interior de ángulo, esto implica que:

\begin{itemize}
  \item $C$ está en el mismo semiplano que $D$ respecto de la $\overleftrightarrow{AB}$.
\end{itemize}

También por hipótesis, el vértice opuesto a $B$, que es $D$, está en el interior
del ángulo en $B$, es decir, en el interior de $\angle ABC$. Esto implica que:

\begin{itemize}
  \item $D$ está en el mismo semiplano que $C$ respecto de la $\overleftrightarrow{AB}$.
\end{itemize}

Juntando ambas afirmaciones, concluimos que $C$ y $D$ están en el mismo semiplano
respecto de la recta que contiene al lado $AB$.

El mismo razonamiento se aplica a los demás lados:

\begin{itemize}
  \item Para el lado $BC$: el vértice opuesto a $B$ es $D$ y el vértice opuesto a
        $C$ es $A$. De que $D$ está en el interior de $\angle ABC$ y $A$ está en
        el interior de $\angle BCD$ se deduce que $A$ y $D$ están en el mismo
        semiplano respecto de la recta $\overleftrightarrow{BC}$.
  \item Para el lado $CD$: del hecho de que $A$ está en el interior de $\angle BCD$
        y $B$ en el interior de $\angle CDA$ se deduce que $A$ y $B$ están en el
        mismo semiplano respecto de la recta $\overleftrightarrow{CD}$.
  \item Para el lado $DA$: del hecho de que $B$ está en el interior de $\angle CDA$
        y $C$ en el interior de $\angle DAB$ se deduce que $B$ y $C$ están en el
        mismo semiplano respecto de la recta $\overleftrightarrow{DA}$.
\end{itemize}

Por lo tanto, para cada lado del cuadrilátero, la recta que lo contiene deja a los
otros dos vértices en el mismo semiplano. Es decir, se cumple la condición (i).
\end{proof}

\subsection{Diagonales}
\begin{proposition}[Caracterización por diagonales]
Sea $ABCD$ un cuadrilátero. Entonces $ABCD$ es convexo en el sentido de la
definición de semiplanos si y sólo si sus diagonales $AC$ y $BD$ se intersectan
en un punto del interior del cuadrilátero.
\end{proposition}

\begin{proof}
($\Rightarrow$) Supongamos que $ABCD$ es convexo según la definición de
semiplanos:
para cada lado, la recta que lo contiene deja a los otros dos vértices en el
mismo semiplano.

Por la proposición de caracterización angular de la convexidad, esto equivale a
que el interior de cada ángulo del cuadrilátero contiene al vértice opuesto. En
particular:
\[
  C \text{ está en el interior de } \angle DAB
  \quad\text{y}\quad
  A \text{ está en el interior de } \angle BCD.
\]

Consideremos la recta $p = \overleftrightarrow{AC}$ y los dos semiplanos que
determina. Probaremos que $B$ y $D$ están en semiplanos opuestos respecto de
$p$.

Del hecho de que $C$ está en el interior de $\angle DAB$ se sigue, por la
definición de interior de ángulo, que:
\begin{itemize}
  \item $C$ está en el mismo semiplano que $D$ respecto de la recta
        $\overleftrightarrow{AB}$, y
  \item $C$ está en el mismo semiplano que $B$ respecto de la recta
        $\overleftrightarrow{AD}$.
\end{itemize}
Análogamente, de que $A$ está en el interior de $\angle BCD$ se obtiene que:
\begin{itemize}
  \item $A$ está en el mismo semiplano que $B$ respecto de la recta
        $\overleftrightarrow{CD}$, y
  \item $A$ está en el mismo semiplano que $D$ respecto de la recta
        $\overleftrightarrow{CB}$.
\end{itemize}

Estas relaciones de semiplanos implican que, al trazar la recta $p$ que une a
$A$ y $C$, los puntos $B$ y $D$ quedan forzosamente en semiplanos distintos de
$p$: si $B$ y $D$ estuvieran en el mismo semiplano respecto de $p$, entonces
uno de los ángulos anteriores no podría contener al vértice opuesto en su
interior (contradiría al menos una de las condiciones “estar en el mismo
semiplano que uno de los lados” en la definición del interior de ángulo).

Por lo tanto, $B$ y $D$ están en semiplanos opuestos respecto de $\overleftrightarrow{AC}$.
Aplicando ahora el teorema de \emph{extremos de segmento en semiplanos opuestos}
a la recta $p$ y al segmento $BD$, concluimos que
\[
  BD \cap p \neq \emptyset.
\]
Es decir, el segmento $BD$ corta a la recta $\overleftrightarrow{AC}$ en un
punto $E$. Como $A$ y $C$ están en $p$ y el cuadrilátero es convexo (la región
determinada por los lados es intersección de semiplanos), el punto de
intersección $E$ pertenece a los segmentos $AC$ y $BD$ y está en el interior
del cuadrilátero. Así,
\[
  E \in AC \cap BD,
\]
es decir, las diagonales se intersectan.

\medskip
($\Leftarrow$) Recíprocamente, supongamos que las diagonales $AC$ y $BD$ se
cortan en un punto $E$ que pertenece al interior del cuadrilátero.

El interior del cuadrilátero se define como la intersección de los cuatro
semiplanos determinados por las rectas que contienen a los lados, tomando en
cada caso el semiplano que contiene “hacia adentro” la figura. El hecho de que
$E$ esté en el interior significa que $E$ pertenece simultáneamente a los
cuatro semiplanos elegidos.

Consideremos, por ejemplo, la recta $r = \overleftrightarrow{AB}$. Como $E$
está en el semiplano “interior” respecto de $r$, y $C$ y $D$ también pertenecen
al interior del cuadrilátero, ninguno de los segmentos $CE$ o $DE$ puede cortar
a $r$ fuera de $A$ o $B$. En particular, el segmento $CD$ no corta a la recta
$r$ en ningún punto. Por el teorema de \emph{segmento contenido en un
semiplano}, se concluye que $C$ y $D$ están en el mismo semiplano respecto de
$\overleftrightarrow{AB}$.

Repitiendo el mismo razonamiento para las rectas que contienen a los otros
lados ($\overleftrightarrow{BC}$, $\overleftrightarrow{CD}$ y
$\overleftrightarrow{DA}$), obtenemos que, para cada lado del cuadrilátero, la
recta que lo contiene deja a los otros dos vértices del mismo lado.

Esto es exactamente la definición de cuadrilátero convexo en términos de
semiplanos. Por lo tanto, $ABCD$ es convexo.
\end{proof}

\subsection{Vértice dentro de un triángulo}

\begin{proposition}[Caracterización por un vértice en un triángulo]
Sea $ABCD$ un cuadrilátero. Entonces $ABCD$ no es convexo si y sólo si
existe un triángulo cuyos vértices son tres vértices del cuadrilátero tal que
el cuarto vértice está en el interior de dicho triángulo.
\end{proposition}

\begin{proof}
($\Leftarrow$) Supongamos primero que existen tres vértices del cuadrilátero
que forman un triángulo y que el cuarto vértice está en el interior de dicho
triángulo. Re-etiquetando si es necesario, podemos suponer que:

\[
\text{el triángulo es } \triangle ABC \quad\text{y el cuarto vértice es } D,
\]
de modo que $D$ está en el interior de $\triangle ABC$.

Por la descripción del interior de un triángulo como intersección de tres
semiplanos, el hecho de que $D$ esté en el interior de $\triangle ABC$
equivale a que $D$ está:

\begin{itemize}
  \item en el mismo semiplano que $C$ respecto de la recta $\overleftrightarrow{AB}$,
  \item en el mismo semiplano que $B$ respecto de la recta $\overleftrightarrow{AC}$,
  \item en el mismo semiplano que $A$ respecto de la recta $\overleftrightarrow{BC}$.
\end{itemize}

Consideremos ahora la recta $r = \overleftrightarrow{CD}$. Como $D$ es interior
al triángulo $\triangle ABC$, cualquier recta que pase por $D$ y por un vértice
del triángulo corta al lado opuesto en un punto interior de ese lado.
En particular, la recta $r$ corta al lado $AB$ en un punto $P$ con
$P \in AB$ y $P \neq A,B$.

Aplicamos el teorema de \emph{extremos de segmento en semiplanos opuestos} a la
recta $r$ y al segmento $AB$: como $AB \cap r = \{P\}$ con $P$ interior al
segmento, se concluye que $A$ y $B$ están en semiplanos opuestos respecto de
$r$.

Es decir, para el lado $CD$ del cuadrilátero, la recta que lo contiene deja a
los otros dos vértices $A$ y $B$ en semiplanos distintos. Por lo tanto, \emph{no}
se cumple la condición de convexidad por semiplanos. Concluimos que $ABCD$ no
es convexo.

\medskip

($\Rightarrow$) Ahora supongamos que $ABCD$ no es convexo. Entonces, por la
definición de convexidad en términos de semiplanos, existe al menos un lado
del cuadrilátero cuyo soporte (la recta que lo contiene) deja a los otros dos
vértices en semiplanos opuestos.

Re-etiquetando los vértices si es necesario, podemos suponer que dicho lado
es $CD$ y que la recta $r = \overleftrightarrow{CD}$ deja a $A$ y $B$ en
semiplanos opuestos. Por el teorema de \emph{extremos de segmento en
semiplanos opuestos}, esto implica que el segmento $AB$ corta a la recta $r$
en un punto $P$ interior al segmento $AB$.

Consideremos ahora el triángulo $\triangle ABC$. La recta $r$ pasa por $C$
y por $P$, punto interior al lado $AB$. De este modo, $r$ divide al triángulo
$\triangle ABC$ en dos partes, y el punto $D$, que está sobre $r$ pero en el
lado “opuesto” respecto de $A$ y $B$, debe quedar en el interior de alguna de
esas dos partes.

Usando nuevamente la descripción del interior de un triángulo como intersección
de semiplanos y los teoremas de segmentación en semiplanos, se verifica que
$D$ cumple las tres condiciones para estar en el interior de $\triangle ABC$:
\begin{itemize}
  \item $D$ está en el mismo semiplano que $C$ respecto de $\overleftrightarrow{AB}$
        (porque $P$ está entre $A$ y $B$ y $C$ está en el mismo lado que $D$
        respecto de $r$),
  \item $D$ está en el mismo semiplano que $B$ respecto de $\overleftrightarrow{AC}$,
  \item $D$ está en el mismo semiplano que $A$ respecto de $\overleftrightarrow{BC}$.
\end{itemize}

Por lo tanto, $D$ está en el interior del triángulo $\triangle ABC$. Es decir,
tres vértices del cuadrilátero (en este caso $A,B,C$) forman un triángulo cuyo
interior contiene al cuarto vértice $D$.

Esto completa la demostración de la equivalencia.
\end{proof}


\end{document}
