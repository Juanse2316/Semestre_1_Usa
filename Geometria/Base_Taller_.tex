\documentclass[12pt,letterpaper]{exam}

\usepackage[left=2cm,top=2cm,right=2cm,bottom=2.5cm]{geometry}
\usepackage{hyperref}
\usepackage[T1]{fontenc}
\usepackage[utf8]{inputenc}
\DeclareUnicodeCharacter{001C}{}
\usepackage[spanish,activeacute]{babel}
\decimalpoint

\usepackage{enumerate}
\usepackage{eurosym}
\usepackage{latexsym,amsmath,amsthm,amssymb,amsfonts,bbm,dsfont}
% \usepackage[mathscr]{euscript}  % <-- QUITADO para evitar choque con mathrsfs
\usepackage{mathrsfs}            % <-- Mantén este si usas \mathscr
% \usepackage{ae,aecompl}         % <-- QUITADO (obsoleto)

% --- TikZ ---
\usepackage{tikz}
\usetikzlibrary{calc,arrows.meta,intersections,positioning}
\tikzset{
  plane/.style={fill=gray!12, draw=black!50},
  line3d/.style={line width=0.9pt},
  pt/.style={circle,fill=black,inner sep=1.4pt},
  label/.style={font=\small}
}

% --- Imágenes y colores ---
\usepackage{graphicx}
\DeclareGraphicsExtensions{.pdf,.png,.jpg}
\usepackage{adjustbox}
\usepackage[dvipsnames]{xcolor}

% --- Caption / subfiguras ---
\usepackage[font=small,labelfont=bf,textfont=it]{caption}
\usepackage{subcaption}  % <-- en lugar de subfigure

\usepackage{enumitem}
\usepackage{float}
\usepackage{multirow}
\usepackage{fancybox}
% \usepackage{mathrsfs}  % ya está arriba

% === RUTAS DE APOYO (opcional) ===
\graphicspath{{../assets/}{../Talleres_fundamentos/}{./}}

%--------------------------------------------------------------------
\newcommand{\base}[1]{\underline{\hspace{#1}}}
%--------------------------------------------------------------------
\newcommand{\uni}{Universidad Sergio Arboleda}
\newcommand{\fac}{\normalsize{Escuela de Ciencias Ex\'actas e Ingenier\'ia}}
\newcommand{\dep}{Matem\'aticas}
\newcommand{\mat}{Geometría Euclidiana} %Materia
\newcommand{\tema}{} %Tipo y Número de Quiz
\newcommand{\autor}{Juan Carlos Ávila M}
%\newcommand{\fecha}{\textbf{Duración}: 100 minutos}
\newcommand{\espacio}[1]{\vspace{#1}}
\renewcommand{\arraystretch}{1.15}

%---------------------------------------------------------------------
\pagestyle{headandfoot}
\footrule
\headrule
\firstpageheader{}{}{}
\firstpagefooter{}{\thepage $\,$ de \numpages}{}
\runningfooter{\uni}{\thepage $\,$ de \numpages}{}

\begin{document}

% ===== Encabezado institucional =====
\begin{tabular}{lr}
    \multirow{2}{*}{\includegraphics[height=1.4cm]{logosergio.png}} &
    {\textbf{\uni}} \\
    & {\textbf{\fac}} \\
    & {\textbf{\dep}} \\
    & {\textbf{\mat \tema}} \\
    & {\textit{\autor}} \\
    & {\textit{}}
\end{tabular}\\
\base{19.5cm}\\
% \textbf{Nombre}: \base{13.2cm} \quad \textit{Calificación}: \base{2cm} \\[6pt]
\textbf{Nombre}: \makebox[11.2cm]{\hfill Juan Sebastian Ladino Mendieta \hfill} \quad 
\textit{Calificaci\'on}: \base{2cm} \\[6pt]

% ===== TALLER 1 =====
\section*{Taller 1 (Postulados de incidencia de la geometr\'ia euclidiana y algunas consecuencias)}

\begin{enumerate}
%   \item Teorema intersección de rectas. Dos rectas diferentes, si se intersecan, lo hacen en exactamente un punto.
 \item Para cada una de las siguientes cuestiones debe decidir si la respuesta a la pregunta es s\'i, no
    o no se sabe. En cada caso, debe argumentar las razones de su respuesta con base en los
    postulados anteriores y los esquemas de razonamiento estudiados:

\begin{enumerate}
  \item \textbf{¿Una recta y un plano pueden ser iguales?} \\
        \textbf{Respuesta: No.} Un plano tiene al menos tres puntos no colineales y en una recta todos son colineales.

  \item \textbf{¿El plano puede ser el espacio?} \\
        \textbf{Respuesta: No.} El espacio tiene al menos cuatro puntos no coplanarios.

  \item \textbf{¿En $\mathscr{L}$ existen exactamente seis rectas?} \\
        \textbf{Respuesta: No se sabe.} Solo se garantiza “al menos” seis; puede haber más.

  \item \textbf{¿En $\mathscr{P}$ existen exactamente cuatro planos?} \\
        \textbf{Respuesta: No se sabe.} Solo se garantiza “al menos” cuatro; puede haber más.

  \item \textbf{¿La unión entre un plano y una recta puede ser un plano?} \\
        \textbf{Respuesta: Depende.} 
        Si $\ell\subset\alpha$, entonces $\alpha\cup\ell=\alpha$ (sí). 
        Si $\ell\not\subset\alpha$, toma $A\in\ell\neq\alpha$ y $B\in\alpha\neq\ell$; la recta $\overleftrightarrow{AB}$ no está contenida en $\alpha\cup\ell$, luego $\alpha\cup\ell$ no es un plano.

  \item \textbf{¿La unión de dos rectas puede ser un punto?} \\
        \textbf{Respuesta: No.} Cada recta tiene al menos dos puntos; la unión no puede tener menos.

  \item \textbf{¿La intersección de una recta y un plano puede ser una recta?} \\
        \textbf{Respuesta: Sí.} Cuando $\ell\subset\alpha$, entonces $\ell\cap\alpha=\ell$; si no, la intersección es un punto.

  \item \textbf{¿La intersección entre dos rectas puede ser una recta?} \\
        \textbf{Respuesta: Sí.} Si son la misma recta, $\ell\cap\ell=\ell$.

  \item \textbf{¿La unión de dos rectas puede ser un plano?} \\
        \textbf{Respuesta: No.} Si $\ell\neq m$, toma $A\in\ell\neq m$ y $C\in m\neq\ell$; la recta $\overleftrightarrow{AC}$ está en el plano que contienen $\ell$ y $m$ pero (salvo $A,C$) no está en $\ell\cup m$. Si $\ell=m$, la unión es una recta, no un plano.
\end{enumerate}

  
  \item \textbf{ Teorema intersección de rectas.} Dos rectas diferentes si se intersecan lo hacen en exacta
mente un punto.

  \begin{center}
      % Teorema 1: Intersección de rectas (tabla de pasos y justificaciones)
      \begin{tabular}{|p{0.62\linewidth}|p{0.33\linewidth}|}
      \hline
      \textbf{Paso} & \textbf{Justificación} \\ \hline
      \vspace{0.2em}
      1. $\ell$ y $m$ dos rectas distintas. & 1. Hipótesis del teorema. \\ \hline
      \vspace{0.2em}
      2. $\ell \cap m \neq \varnothing$. & 2. Definición de “se intersecan”. \\ \hline
      \vspace{0.2em}
      3. Supongamos que existen $A \neq B$ y $A, B \in \ell \cap m$. & 3. Negación de la conclusión (prueba por contradicción). \\ \hline
      \vspace{0.2em}
      4. $A, B \in \ell$. & 4. Definición de intersección / contención de conjuntos (3). \\ \hline
      \vspace{0.2em}
      5. $A, B \in m$. & 5. Definición de intersección / contención de conjuntos (3). \\ \hline
      \vspace{0.2em}
      6. $\overleftrightarrow{AB} = \ell$. & 6. Postulado de la recta (unicidad con $A\neq B$ y 4). \\ \hline
      \vspace{0.2em}
      7. $\overleftrightarrow{AB} = m$. & 7. Postulado de la recta (unicidad con $A\neq B$ y 5). \\ \hline
      \vspace{0.2em}
      8. $\ell = m$. & 8. Unicidad de la recta por (6) y (7). \\ \hline
      \vspace{0.2em}
      9. $A = B$. & 9. Principio de prueba indirecta: (1) contradice (8) bajo la suposición (3). \\ \hline
      \vspace{0.2em}
      10. $\ell \cap m \neq \varnothing$. & 10. Hipótesis reiterada (2). \\ \hline
      \end{tabular}
\end{center}


  \item A continuación aparece el enunciado de algunos teoremas junto con los pasos de la demostración de cada uno. Los pasos están en desorden y sin justificar; por tanto, la tarea consiste en
  ordenar los pasos de las demostraciones y escribir las justificaciones respectivas:

% ================== LISTA DE TRES TEOREMAS ==================
\begin{enumerate}[label=\alph*)]
%______________________Teorema intersección recta--plano_______________________________________%
\item \textbf{Teorema intersecci\'on recta--plano.} 
Si una recta interseca a un plano que no la contiene, entonces la intersecci\'on consiste en un \'unico punto.

%______________________Imagen de Teorema intersección recta--plano_______________________________________%

\begin{figure}[H]
\centering
\begin{adjustbox}{max width=0.7\textwidth}
\begin{tikzpicture}[scale=0.95]
  % Estilos locales
  \tikzset{
    plane/.style={fill=gray!12, draw=black!50},
    line3d/.style={line width=0.9pt},
    pt/.style={circle,fill=black,inner sep=1.6pt},
    label/.style={font=\small}
  }
  % Plano alpha (paralelogramo)
  \coordinate (P1) at (0,0);
  \coordinate (P2) at (4,1);
  \coordinate (P3) at (6,0);
  \coordinate (P4) at (2,-1);
  \draw[plane] (P1)--(P2)--(P3)--(P4)--cycle;
  \node[label] at (1.0,0.75) {$\alpha$};

  % Recta l (no contenida) que atraviesa el plano
  \draw[line3d] (-0.6,-1.6) -- (6.7,1.9) node[above right] {$\ell$};

  % Punto de intersección
  \coordinate (I) at (3.15,0.15);
  \node[pt,label=above left:$I$] at (I) {};
  \node[label] at (5.2,1.5) {$\ell \cap \alpha = \{I\}$};
\end{tikzpicture}
\end{adjustbox}
\caption{Intersecci\'on recta--plano con $\ell\not\subset\alpha$: la intersecci\'on es un \'unico punto.}
\end{figure}

%______________________Imagen de Teorema intersección recta--plano_______________________________________%

%______________________tabla de Teorema intersección recta--plano_______________________________________%

\begin{center}
      \begin{tabular}{|p{0.7\linewidth}|p{0.33\linewidth}|}
      \hline
      \textbf{Paso} & \textbf{Justificación} \\ \hline
      % \vspace{0.2em}
      1. $\ell$ una recta. & 1. Hipótesis del teorema. \\ \hline
      % \vspace{0.2em}
      2. $\alpha$ un plano. & 2. Hipótesis del teorema. \\ \hline
      % \vspace{0.2em}
      3. $\ell \nsubseteq \alpha$. & 3. Hipótesis: “que no la contiene”. \\ \hline
      % \vspace{0.2em}
      4. $\ell \cap \alpha \neq \varnothing$. & 4. D. de “interseca”. \\ \hline
      % \vspace{0.2em}
      5. Sean $A$ y $B$ dos puntos distintos de modo que $A,B \in \ell \cap \alpha$. & 5. Negación de la conclusión (prueba por contradicción). \\ \hline
      % \vspace{0.2em}
      6. $A, B \in \ell$. & D. de intersección / contención de conjuntos (5). \\ \hline
      \vspace{0.001em}
      7. Existe $\overleftrightarrow{AB}$. & 7. P de la recta (con $A\neq B$ de 5). \\ \hline
      \vspace{0.001em}
      8. $\overleftrightarrow{AB} = \ell$. & 8. Unicidad de la recta por (1, 6, 7). \\ \hline
      % \vspace{0.001em}
      9. $\ell \subseteq \alpha$. & 9. P. de la llaneza del plano: de (5) $A,B\in\alpha$ y por (8) la recta que los contiene es $\ell$. \\ \hline
      % \vspace{0.2em}
      10. $A = B$. & 10. Principio de prueba indirecta (3, 9, 5). \\ \hline
      \end{tabular}
\end{center}
%______________________tabla de Teorema intersección recta--plano_______________________________________%
\vspace{1em}

%______________________Teorema recta--punto--plano_______________________________________%
\item \textbf{Teorema recta--punto--plano.}
Dada una recta $\ell$ y un punto $P\notin \ell$, existe exactamente un plano que contiene a $\ell$ y $P$.

%__________________ Imagen de Teorema recta--punto--plano_______________________________________%

\begin{figure}[h]
\centering
\begin{tikzpicture}
\coordinate (P1) at (-5,-0.6);
\coordinate (P2) at (-1, 1.6);
\coordinate (P3) at ( 4, 1.3);
\coordinate (P4) at ( 0,-1.2);

\fill[orange!25] (P1)--(P2)--(P3)--(P4)--cycle;
\draw[orange!70, line width=1] (P1)--(P2)--(P3)--(P4)--cycle;


\coordinate (A) at (-3.6,-0.20);
\coordinate (B) at ( 2.7, 0.95);
\draw[very thick] (A)--(B) node[midway, above] {$f$};

\fill[blue] (A) circle (2.2pt) node[left] {$A$};
\fill[blue] (B) circle (2.2pt) node[above right] {$B$};
\fill[blue] (1.2,-0.25) circle (2.2pt) node[right] {$P$};
\end{tikzpicture}
\caption{Dada una recta $\ell$ y un punto $P\notin\ell$, existe un \'unico plano $\alpha$ con $\ell\subset\alpha$ y $P\in\alpha$.}
\end{figure}


%_________________________________________________________%



%__________________tabla de Teorema recta--punto--plano_______________________________________%

\begin{center}
      \begin{tabular}{|p{0.62\linewidth}|p{0.33\linewidth}|}
      \hline
      \textbf{Paso} & \textbf{Justificación} \\ \hline
      1. $\ell$ una recta. & 1. Hipótesis del teorema. \\ \hline
      2. $A, B$ puntos distintos en $\ell$. & 2. P. mínima cantidad de puntos aplicado a (1). \\ \hline
      3. $P$ un punto. & 3. Hipótesis del teorema. \\ \hline
      4. $P \notin \ell$. & 4. Hipótesis del teorema. \\ \hline
      5. $A, B$ y $P$ son no colineales. & 5. Definición de colinealidad y unicidad de la recta por (2, 4). \\ \hline
      6. Existe un único plano $\alpha$ que contiene a $A, B$ y $P$. & 6. P. plano (tres puntos no colineales) a partir de (5). \\ \hline
      \vspace{0.001em}
      7. $\overleftrightarrow{AB} \subseteq \alpha$. & 7. P. llaneza del plano: de (6) $A,B\in\alpha$; la recta por $A,B$ (única) está contenida en $\alpha$. \\ \hline
      \vspace{0.001em}
      8. $\{P\} \cup \overleftrightarrow{AB} \subseteq \alpha$. & 8. D. contención de conjuntos usando (6, 7). \\ \hline
      \end{tabular}
\end{center}
%__________________tabla de Teorema recta--punto--plano_______________________________________%
\vspace{1em}

%______________________Teorema rectas intersecantes--plano_______________________________________%
\item \textbf{Teorema rectas intersecantes--plano.}
Si dos rectas distintas $\ell,m$ se intersecan, entonces ambas est\'an contenidas en exactamente un plano.

%______________________Image Teorema rectas intersecantes--plano_______________________________________%

\begin{figure}[h!]
\centering
\begin{tikzpicture}[scale=1]

% Definir puntos
\coordinate (A) at (2,3.5);
\coordinate (B) at (0.5,1.5);
\coordinate (C) at (3.5,1.5);

% Dibujar plano como un triángulo
\filldraw[fill=orange!20, draw=orange] (A) -- (B) -- (C) -- cycle;
\node at (2,2.3) {$\alpha$};

% Dibujar las rectas
\draw[thick] (-0.14,0.64) -- (3.06,4.92) node[right] {$\ell$};
\draw[thick] (4.21,0.55) -- (1.14,4.65) node[right] {$m$};

% Etiquetas de puntos
\fill (A) circle (2pt) node[above] {$A$};
\fill (B) circle (2pt) node[left] {$B$};
\fill (C) circle (2pt) node[right] {$C$};

\end{tikzpicture}
\caption{Figura 3: Dos rectas distintas $\ell$ y $m$ que se intersecan en $A$ yacen en un único plano $\alpha$.}
\end{figure}


%_______________________________________________________________________________________________________%


%______________________Tabla de Teorema rectas intersecantes--plano_______________________________________%
\begin{center}
      \begin{tabular}{|p{0.45\linewidth}|p{0.45\linewidth}|}
      \hline
      \textbf{Enunciados} & \textbf{Justificación} \\
      \hline
      1. $\ell$ y $m$ dos rectas distintas. & 1. Hipótesis \\
      \hline
      2. $\ell \cap m \neq \varnothing$. & 2. Hipótesis \\
      \hline
      3. $\ell \cap m = \{A\}$. & 3. T. intersección de rectas (1,2) \\
      \hline
      4. $A \in \ell$. & 4. D. intersección (3) \\
      \hline
      5. $A \in m$. & 5. D. intersección (3) \\
      \hline
      6. $B \in \ell$ y $B \neq A$. & 6. P. mínima cantidad de puntos (4) \\
      \hline
      7. $B \in \ell$. & 7. Eliminación de $\land$ (6) \\
      \hline
      8. $C \in m$ y $C \neq A$. & 8. P. mínima cantidad de puntos (5) \\
      \hline
      9. $C \in m$. & 9. Eliminación de $\land$ (8) \\
      \hline
      10. $A, B$ y $C$ puntos no colineales. & 10. Si fueran colineales $\Rightarrow$ dos puntos comunes $\Rightarrow \ell=m$, contra (1,3) \\
      \hline
      11. Sea $\alpha$ el único plano que contiene a $A, B$ y $C$. & 11. P. plano (10) \\
      \hline
      \vspace{0.001em}
      12. $\overleftrightarrow{AB} = \ell$. & 12. P. recta (unicidad) (4,6,7) \\
      \hline
      \vspace{0.001em}
      13. $\overleftrightarrow{AC} = m$. & 13. P. recta (unicidad) (5,8,9) \\
      \hline
      \vspace{0.001em}
      14. $\overleftrightarrow{AB} \subseteq \alpha$. & 14. P. llaneza del plano (11) \\
      \hline
      \vspace{0.001em}
      15. $\overleftrightarrow{AC} \subseteq \alpha$. & 15. P. llaneza del plano (11) \\
      \hline
      \vspace{0.001em}
      16. $\overleftrightarrow{AB} \subseteq \alpha \;\; \text{y} \;\; \overleftrightarrow{AC} \subseteq \alpha$. & 16. Introducción de $\land$ (14,15) \\
      \hline
      \end{tabular}
\end{center}
%___________________________________________________________________________________________________________



\end{enumerate} % cierra lista a), b), c)

\end{enumerate} % cierra lista principal

\end{document}
