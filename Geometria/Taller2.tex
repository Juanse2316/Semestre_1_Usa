\documentclass[12pt,letterpaper]{exam}

\usepackage[left=2cm,top=2cm,right=2cm,bottom=2.5cm]{geometry}
\usepackage{hyperref}
\usepackage[T1]{fontenc}
\usepackage[utf8]{inputenc}
\DeclareUnicodeCharacter{001C}{}
\usepackage[spanish,activeacute]{babel}
\decimalpoint

\usepackage{enumerate}
\usepackage{eurosym}
\usepackage{latexsym,amsmath,amsthm,amssymb,amsfonts,bbm,dsfont}
\usepackage{mathrsfs}

% --- TikZ ---
\usepackage{tikz}
\usetikzlibrary{calc,arrows.meta,intersections,positioning}
\tikzset{
  plane/.style={fill=gray!12, draw=black!50},
  line3d/.style={line width=0.9pt},
  pt/.style={circle,fill=black,inner sep=1.4pt},
  label/.style={font=\small}
}

% --- Imágenes y colores ---
\usepackage{graphicx}
\DeclareGraphicsExtensions{.pdf,.png,.jpg}
\usepackage{adjustbox}
\usepackage[dvipsnames]{xcolor}

% --- Caption / subfiguras ---
\usepackage[font=small,labelfont=bf,textfont=it]{caption}
\usepackage{subcaption}

\usepackage{enumitem}
\usepackage{float}
\usepackage{multirow}
\usepackage{fancybox}

% --- Tablas avanzadas ---
\usepackage{tabularx}
\usepackage{array}
\usepackage{booktabs}

% Definición de columnas tipo L y R para tabularx/ltablex
\newcolumntype{L}{>{\raggedright\arraybackslash}X}
\newcolumntype{R}{>{\raggedright\arraybackslash}X}

% --- Tablas X que se parten entre páginas ---
\usepackage{ltablex}   % combina tabularx + longtable
\keepXColumns          % mantiene anchos X constantes entre páginas
\setlength\LTleft{0pt}
\setlength\LTright{0pt}

% ===== Helper para filas de ancho completo que envuelven texto =====
\newcommand{\fullrow}[1]{%
  \multicolumn{2}{@{}>{\raggedright\arraybackslash}p{\linewidth}@{}}{#1}%
}

% (opcional) margen de emergencia para evitar overfull hboxes
\emergencystretch=2em

% === RUTAS DE APOYO (opcional) ===
\graphicspath{{../assets/}{../Talleres_fundamentos/}{./}}

%--------------------------------------------------------------------
\newcommand{\base}[1]{\underline{\hspace{#1}}}
\newcommand{\uni}{Universidad Sergio Arboleda}
\newcommand{\fac}{\normalsize{Escuela de Ciencias Ex\'actas e Ingenier\'ia}}
\newcommand{\dep}{Matem\'aticas}
\newcommand{\mat}{Geometría Euclidiana}
\newcommand{\tema}{}
\newcommand{\autor}{Juan Carlos Ávila M}
\newcommand{\espacio}[1]{\vspace{#1}}
\renewcommand{\arraystretch}{1.15}

%---------------------------------------------------------------------
\pagestyle{headandfoot}
\footrule
\headrule
\firstpageheader{}{}{}
\firstpagefooter{}{\thepage $\,$ de \numpages}{}
\runningfooter{\uni}{\thepage $\,$ de \numpages}{}

\begin{document}

% ===== Encabezado institucional =====
\begin{tabular}{lr}
    \multirow{2}{*}{\includegraphics[height=1.4cm]{logosergio.png}} &
    {\textbf{\uni}} \\
    & {\textbf{\fac}} \\
    & {\textbf{\dep}} \\
    & {\textbf{\mat \tema}} \\
    & {\textit{\autor}} \\
    & {\textit{}}
\end{tabular}\\
\base{19.5cm}\\
\textbf{Nombre}: \makebox[11.2cm]{\hfill Juan Sebastian Ladino Mendieta \hfill} \quad 
\textit{Calificaci\'on}: \base{2cm} \\[6pt]


%===========================================================
% TEORMA Z
%===========================================================
\section*{Teorema Z}
\textbf{Enunciado.} Si $D\in\operatorname{int}\angle BAC$ y $F-A-C$, entonces $F$ y $B$ están del mismo lado de la $\overleftrightarrow{AD}$.

\subsection*{Demostración (doble columna)}

% ------ TABLA PARTIBLE CON LTABLEX ------
\begin{tabularx}{\linewidth}{@{}L R@{}}
\toprule
\textbf{Afirmación} & \textbf{Razón} \\
\midrule
\endfirsthead
\toprule
\textbf{Afirmación} & \textbf{Razón \,(cont.)} \\
\midrule
\endhead
\midrule
\multicolumn{2}{r}{\small Continúa en la siguiente página} \\
\bottomrule
\endfoot
\bottomrule
\endlastfoot

1.\; $\angle BAC$ & Datos: $A,B,C$ no colineales. \\
2.\; $F-A-C$ & Hipótesis. \\
3.\; $D\in\operatorname{int}\angle BAC$ & Hipótesis. \\
4.\; Sea $G$ tal que $G-A-D$. & Construcción sobre $\overleftrightarrow{AD}$. \\
5.\; $\overleftrightarrow{AD}=\overrightarrow{AD}\cup\overrightarrow{AG}$. & Propiedad de recta como unión de semirrectas opuestas. \\
\fullrow{\emph{Primera parte}}\\
6.\; $F$ y $C$ están del mismo lado de $\overleftrightarrow{AB}$. & Teorema de interestancia (mismo semiplano) aplicado a $F-A-C$. \\
7.\; $\overrightarrow{BF}-\{B\}\subseteq S_{\overleftrightarrow{AB},\,C}$. & Teorema de la semirrecta. \\
8.\; Ningún punto de $FB$ está en $\operatorname{int}\angle BAC$. & \textit{Ver Justificación (Z-8).} \\
9.\; $\overline{FB}$ no tiene puntos de $\overrightarrow{AD}$. & (3) + ``Semirrecta en el interior del ángulo'': $\overrightarrow{AD}-\{A\}\subseteq \operatorname{int}\angle BAC$; usar (8). \\
10.\; $\overrightarrow{AD}\cap \overline{FB}=\varnothing$. & Reescritura de (9). \\
\fullrow{\emph{Segunda parte}}\\
11.\; $\overrightarrow{FB}-\{F\}\subseteq S_{\overleftrightarrow{AC},\,B}$. & Teorema de la semirrecta. \\
12.\; $\overline{FB}-\{F\}\subseteq S_{\overleftrightarrow{AC},\,B}$. & \textit{Ver Justificación (Z-12).} \\
13.\; $\overrightarrow{AG}-\{A\}\subseteq S_{\overleftrightarrow{AC},\,\neg B}$. & Teorema de la semirrecta 2 (si $D\in\operatorname{int}\angle BAC$ y $G-A-D$). \\
14.\; $\overline{FB}$ no tiene puntos de $\overrightarrow{AG}$. & \textit{Ver Justificación (Z-14).} \\
15.\; $\overline{FB}\cap\overrightarrow{AG}=\varnothing$. & Reescritura de (14). \\
16.\; $\overleftrightarrow{AD}\cap FB=\varnothing$. & (5), (10) y (15): $FB\cap(\overrightarrow{AD}\cup\overrightarrow{AG})=(FB\cap\overrightarrow{AD})\cup(FB\cap\overrightarrow{AG})=\varnothing$. \\
\midrule
\fullrow{\textbf{Conclusión.}\;
Por ``segmento contenido en un semiplano'',
$\overline{FB}\cap\overleftrightarrow{AD}=\varnothing$ implica que
$F$ y $B$ están del mismo lado de $\overleftrightarrow{AD}$. \qed}\\

\end{tabularx}

\subsubsection*{Justificaciones Teorema Z}
\begin{itemize}[leftmargin=2.2em]
  \item \textbf{(Z-8)} \; Por definición, $X\in\operatorname{int}\angle BAC$ ssi: (i) $X$ está del mismo lado de $\overleftrightarrow{AB}$ que $C$ y (ii) del mismo lado de $\overleftrightarrow{CB}$ que $A$.  
  Por (7), todo $X\in\overrightarrow{BF}-\{B\}$ satisface (i). Sin embargo, como $B\in\overleftrightarrow{CB}$ y $F-A-C$, el teorema de interestancia (lados opuestos) respecto de $\overleftrightarrow{CB}$ muestra que esos $X$ están del lado opuesto al de $A$, con lo cual (ii) falla. Por tanto, ningún punto de $FB$ está en el interior.
  \item \textbf{(Z-12)} \; $FB-\{F\}\subseteq \overrightarrow{FB}-\{F\}$ (el segmento está contenido en la semirrecta que parte de $F$ y pasa por $B$). Combinando con (11) se obtiene la contención.
  \item \textbf{(Z-14)} \; Por (12), $FB-\{F\}\subseteq S_{\overleftrightarrow{AC},\,B}$, mientras que por (13), $\overrightarrow{AG}-\{A\}\subseteq S_{\overleftrightarrow{AC},\,\neg B}$. Estos semiplanos son opuestos y disjuntos; por tanto no hay intersección entre $FB$ y $\overrightarrow{AG}$ (salvo posibles extremos, que aquí no aplican).
\end{itemize}


%===========================================================
% TEORMA DE LA BARRA CRUZADA
%===========================================================
\section*{Teorema de la barra cruzada}
\textbf{Enunciado.} Si $D\in\operatorname{int}\angle BAC$, entonces $\overrightarrow{AD}$ interseca a $BC$.

\subsection*{Demostración (doble columna)}

% ------ TABLA PARTIBLE CON LTABLEX ------
\begin{tabularx}{\linewidth}{@{}L R@{}}
\toprule
\textbf{Afirmación} & \textbf{Razón} \\
\midrule
\endfirsthead
\toprule
\textbf{Afirmación} & \textbf{Razón \,(cont.)} \\
\midrule
\endhead
\midrule
\multicolumn{2}{r}{\small Continúa en la siguiente página} \\
\bottomrule
\endfoot
\bottomrule
\endlastfoot

1.\; $\angle BAC$. & $A,B,C$ no colineales (triángulo). \\
2.\; $D\in\operatorname{int}\angle BAC$. & Hipótesis. \\
3.\; Supongamos que $\overrightarrow{AD}\cap \overline{BC}=\varnothing$. & Suposición para contradicción. \\
4.\; $B$ y $C$ están del mismo lado de $\overleftrightarrow{AD}$. & \textit{Ver Justificación (BC-4).} \\
5.\; Sea $F$ tal que $F-A-C$. & Construcción sobre $\overleftrightarrow{AC}$. \\
6.\; $F$ y $B$ están del mismo lado de $\overleftrightarrow{AD}$. & Teorema Z (usando (2) y (5)). \\
7.\; $F$ y $C$ están del mismo lado de $\overleftrightarrow{AD}$. & Interestancia (mismo semiplano) aplicado a $F-A-C$. \\
8.\; $\overline{FC}\cap \overleftrightarrow{AD}=\varnothing$. & ``Segmento contenido en un semiplano'' con extremos (6) y (7). \\
9.\; $\overline{FC}\cap \overleftrightarrow{AD}=\{A\}$. & Porque $F-A-C$, $A\in\overline{FC}$ y $A\in\overleftrightarrow{AD}$; único punto común. \\
10.\; $\overrightarrow{AD}\cap \overline{BC}\neq\varnothing$. & (8)–(9) contradicen (3). Por tanto, existe la intersección. \\
\midrule
\fullrow{\textbf{Conclusión.}\; $\overrightarrow{AD}$ corta al segmento $BC$. \qed}\\

\end{tabularx}

\subsubsection*{Justificación Teorema de la barra cruzada}
\begin{itemize}[leftmargin=2.2em]
  \item \textbf{(BC-4)} \; Si $B$ y $C$ estuvieran en lados opuestos de $\overleftrightarrow{AD}$, entonces por ``extremos de segmento en semiplanos opuestos'' el segmento $\overline{BC}$ cortaría a $\overleftrightarrow{AD}$. Ese punto de corte, al estar en el triángulo $ABC$, no puede pertenecer a $\overrightarrow{AG}$ (usando que los puntos de $BC$ están en $\operatorname{int}\angle BAC$ y la Semirrecta~2 sitúa $\overrightarrow{AG}-\{A\}$ en el semiplano opuesto); por tanto caería en $\overrightarrow{AD}$, contradiciendo la suposición (3). De ahí que $B$ y $C$ queden del mismo lado.
\end{itemize}

\end{document}
