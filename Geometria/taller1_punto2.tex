\documentclass[12pt,letterpaper]{exam}

\usepackage[left=2cm,top=2cm,right=2cm,bottom=2.5cm]{geometry}
\usepackage{hyperref}
\usepackage[T1]{fontenc}
\usepackage[utf8]{inputenc}
\DeclareUnicodeCharacter{001C}{}
\usepackage[spanish,activeacute]{babel}
\decimalpoint

\usepackage{enumerate}
\usepackage{eurosym}
\usepackage{latexsym,amsmath,amsthm,amssymb,amsfonts,bbm,dsfont}
% \usepackage[mathscr]{euscript}  % <-- QUITADO para evitar choque con mathrsfs
\usepackage{mathrsfs}            % <-- Mantén este si usas \mathscr
% \usepackage{ae,aecompl}         % <-- QUITADO (obsoleto)

% --- TikZ ---
\usepackage{tikz}
\usetikzlibrary{calc,arrows.meta,intersections,positioning}
\tikzset{
  plane/.style={fill=gray!12, draw=black!50},
  line3d/.style={line width=0.9pt},
  pt/.style={circle,fill=black,inner sep=1.4pt},
  label/.style={font=\small}
}

% --- Imágenes y colores ---
\usepackage{graphicx}
\DeclareGraphicsExtensions{.pdf,.png,.jpg}
\usepackage{adjustbox}
\usepackage[dvipsnames]{xcolor}

% --- Caption / subfiguras ---
\usepackage[font=small,labelfont=bf,textfont=it]{caption}
\usepackage{subcaption}  % <-- en lugar de subfigure

\usepackage{enumitem}
\usepackage{float}
\usepackage{multirow}
\usepackage{fancybox}
% \usepackage{mathrsfs}  % ya está arriba

% === RUTAS DE APOYO (opcional) ===
\graphicspath{{../assets/}{../Talleres_fundamentos/}{./}}

%--------------------------------------------------------------------
\newcommand{\base}[1]{\underline{\hspace{#1}}}
%--------------------------------------------------------------------
\newcommand{\uni}{Universidad Sergio Arboleda}
\newcommand{\fac}{\normalsize{Escuela de Ciencias Ex\'actas e Ingenier\'ia}}
\newcommand{\dep}{Matem\'aticas}
\newcommand{\mat}{Geometría Euclidiana} %Materia
\newcommand{\tema}{} %Tipo y Número de Quiz
\newcommand{\autor}{Juan Carlos Ávila M}
%\newcommand{\fecha}{\textbf{Duración}: 100 minutos}
\newcommand{\espacio}[1]{\vspace{#1}}
\renewcommand{\arraystretch}{1.15}

%---------------------------------------------------------------------
\pagestyle{headandfoot}
\footrule
\headrule
\firstpageheader{}{}{}
\firstpagefooter{}{\thepage $\,$ de \numpages}{}
\runningfooter{\uni}{\thepage $\,$ de \numpages}{}

\begin{document}

% ===== Encabezado institucional =====
\begin{tabular}{lr}
    \multirow{2}{*}{\includegraphics[height=1.4cm]{logosergio.png}} &
    {\textbf{\uni}} \\
    & {\textbf{\fac}} \\
    & {\textbf{\dep}} \\
    & {\textbf{\mat \tema}} \\
    & {\textit{\autor}} \\
    & {\textit{}}
\end{tabular}\\
\base{19.5cm}\\
% \textbf{Nombre}: \base{13.2cm} \quad \textit{Calificación}: \base{2cm} \\[6pt]
\textbf{Nombre}: \makebox[11.2cm]{\hfill Juan Sebastian Ladino Mendieta \hfill} \quad 
\textit{Calificaci\'on}: \base{2cm} \\[6pt]

% ===== TALLER 1 =====
\section*{Taller 1 (Postulados de incidencia de la geometr\'ia euclidiana y algunas consecuencias)}

A continuación aparece el enunciado de algunos teoremas junto con los pasos de la demostración de cada uno. Los pasos están en desorden y sin justificar; por tanto, la tarea consiste en
ordenar los pasos de las demostraciones y escribir las justificaciones respectivas:

% ================== LISTA DE TRES TEOREMAS ==================
\begin{enumerate}[label=\alph*)]

%______________________Teorema recta--punto--plano_______________________________________%
\item \textbf{Teorema recta--punto--plano.}
Dada una recta $\ell$ y un punto $P\notin \ell$, existe exactamente un plano que contiene a $\ell$ y $P$.

%__________________ Imagen de Teorema recta--punto--plano_______________________________________%

\begin{figure}[H]
\centering
\begin{tikzpicture}
\coordinate (P1) at (-5,-0.6);
\coordinate (P2) at (-1, 1.6);
\coordinate (P3) at ( 4, 1.3);
\coordinate (P4) at ( 0,-1.2);

\fill[orange!25] (P1)--(P2)--(P3)--(P4)--cycle;
\draw[orange!70, line width=1] (P1)--(P2)--(P3)--(P4)--cycle;

\coordinate (A) at (-3.6,-0.20);
\coordinate (B) at ( 2.7, 0.95);
\draw[very thick] (A)--(B) node[midway, above] {$\ell$};

\fill[blue] (A) circle (2.2pt) node[left] {$A$};
\fill[blue] (B) circle (2.2pt) node[above right] {$B$};
\fill[blue] (1.2,-0.25) circle (2.2pt) node[right] {$P$};
\end{tikzpicture}
\caption{Dada una recta $\ell$ y un punto $P\notin\ell$, existe un \'unico plano $\alpha$ con $\ell\subset\alpha$ y $P\in\alpha$.}
\end{figure}

%__________________tabla de Teorema recta--punto--plano_______________________________________%

\begin{center}
\begin{tabular}{|p{0.62\linewidth}|p{0.33\linewidth}|}
\hline
\textbf{Paso} & \textbf{Justificación} \\ \hline
1. $\ell$ una recta. & 1. Hipótesis del teorema. \\ \hline
2. $A, B$ puntos distintos en $\ell$. & 2. P. mínima cantidad de puntos aplicado a (1). \\ \hline
3. $P$ un punto. & 3. Hipótesis del teorema. \\ \hline
4. $P \notin \ell$. & 4. Hipótesis del teorema. \\ \hline
5. $A, B$ y $P$ son no colineales. & 5. Definición de colinealidad y unicidad de la recta por (2, 4). \\ \hline
6. Existe un único plano $\alpha$ que contiene a $A, B$ y $P$. & 6. P. plano (tres puntos no colineales) a partir de (5). \\ \hline
\vspace{0.001em}
7. $\overleftrightarrow{AB} \subseteq \alpha$. & 7. P. llaneza del plano: de (6) $A,B\in\alpha$; la recta por $A,B$ (única) está contenida en $\alpha$. \\ \hline
\vspace{0.001em}
8. $\{P\} \cup \overleftrightarrow{AB} \subseteq \alpha$. & 8. D. contención de conjuntos usando (6, 7). \\ \hline
\end{tabular}
\end{center}
\vspace{1em}



\end{enumerate} % cierra lista principal

\end{document}
