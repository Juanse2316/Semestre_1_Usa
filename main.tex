\documentclass{article}

\usepackage[spanish]{babel}
\usepackage{amsmath, amssymb, amsthm}
\usepackage{tabularx}
\usepackage{array}
\usepackage{setspace}
\usepackage{fancyhdr}
\usepackage{tikz}
\usepackage{pgfplots}
\usepackage{needspace}
\usepackage{booktabs}
\usepackage{graphicx}
\usepackage{subcaption}
\usepackage[a4paper, left=1in, right=1in, top=1in, bottom=1in]{geometry}
\usepackage[section]{placeins} % evita que los flotantes salten de sección
\usepackage{siunitx}
\usepackage{etoolbox}
\usepackage{caption}
\usepackage{float}

% --- Configuración general ---
\pgfplotsset{compat=1.18}
\sisetup{output-decimal-marker = {,}} % <- tu preferencia que sí compila en más versiones
\graphicspath{{../rectangulos_area24_imagenes/}}

% Interlineado general
\doublespacing

% Compacidad: menos huecos alrededor de figuras/tablas y no estirar páginas
\raggedbottom
\setlength{\textfloatsep}{10pt plus 2pt minus 2pt}
\setlength{\floatsep}{8pt plus 2pt minus 2pt}
\setlength{\intextsep}{8pt plus 2pt minus 2pt}

% Interlineado simple y captions compactas dentro de TODAS las figuras y tablas
\AtBeginEnvironment{figure}{\singlespacing}
\AtBeginEnvironment{table}{\singlespacing}
\captionsetup{aboveskip=6pt,belowskip=0pt}
\captionsetup[subfigure]{justification=centering,singlelinecheck=false,aboveskip=2pt,belowskip=2pt}

% Encabezados/pies
\pagestyle{fancy}
\lhead{Juan Ladino}
\rhead{Solución de Talleres}
\cfoot{\thepage}

\begin{document}

\title{Soluciones de los Talleres}
\author{Juan Ladino}
\date{\today}
\maketitle

% TOC más compacto (sin doble interlineado)
{\tableofcontents}

%__________________________________taller 2__________________________________%
\clearpage
\section{Taller 2}

\subsection{Punto 1: Demostración algebraica de que $0.\overline{9} = 1$}

\vspace{1em}
% Reducimos el interlineado solo en la tabla para que quede compacto
\begingroup\setstretch{1.15}
\begin{tabularx}{\textwidth}{@{}>{\raggedright\arraybackslash}p{.55\textwidth} X@{}}
 $\displaystyle x = 0.\overline{9}$.& Multiplicamos por 10.\\[1ex]
$\displaystyle 10x = 9.\overline{9}$ & \\

$\displaystyle 10x - x = 9.\overline{9} - 0.\overline{9}$ &
Restamos la ecuación original. \\[1ex]

$\displaystyle 9x = 9 \;\Rightarrow\; x = \frac{9}{9}$ &
Obtenemos: \\[1ex]
$\displaystyle x = 1$ &
Por lo tanto, $0.\overline{9}=1$. \\[1ex]
\end{tabularx}
\endgroup

\subsection{Punto 2: Comparación de fracciones de forma algebraica y gráfica}

Para comparar $\frac{2}{5}$ y $\frac{3}{7}$, se presentan dos métodos: primero la multiplicación cruzada y luego un método gráfico con diagrama de barras.

\vspace{1em}
\subsubsection{Multiplicación cruzada}
\vspace{0.25em}
\begingroup\setstretch{1.15}
\begin{tabularx}{\textwidth}{@{}>{\raggedright\arraybackslash}p{.55\textwidth} X@{}}
$\displaystyle 2\times7=14,\quad 5\times3=15$ & Primero multiplicamos cruzado. \\[1ex]
$\displaystyle 14<15$ & Comparamos los resultados. \\[1ex]
$\displaystyle \frac{2}{5}< \frac{3}{7}$ & Concluimos que $\tfrac{2}{5}$ es menor que $\tfrac{3}{7}$. \\[1ex]
\end{tabularx}
\endgroup

\subsubsection{Gráfica de las fracciones}

En la gráfica se aprecia que $\frac{3}{7}\approx\num{0.42857}$ es mayor que $\frac{2}{5}=\num{0.4}$.

\begin{figure}[H]
\centering
\begin{tikzpicture}
\begin{axis}[
  width=\linewidth, height=0.34\textheight,
  ybar, ymin=0, ymax=1, ytick distance=0.2,
  bar width=18pt,
  nodes near coords, nodes near coords align={vertical},
  enlarge x limits=0.4,
  axis lines=left, grid=major, ylabel={Valor de la fracción},
  xtick={1,2}, xticklabels={$\,\frac{2}{5}\,$,$\,\frac{3}{7}\,$}
]
  \addplot coordinates {(1,0.4) (2,0.42857)};
\end{axis}
\end{tikzpicture}
\caption{Comparación gráfica entre $\frac{2}{5}$ y $\frac{3}{7}$.}
\end{figure}

% Reservamos buen espacio para que el siguiente bloque no quede cortado feo
\Needspace{12\baselineskip}
\subsection{Demostración del cálculo del 10\% del 30\% de una nota}

Sea $N$ la nota total.

\vspace{1em}
\begingroup\setstretch{1.15}
\begin{tabularx}{\textwidth}{@{}>{\raggedright\arraybackslash}p{.62\textwidth} X@{}}
$\displaystyle \frac{30}{100} \cdot N$ &
Calculamos el 30\% de la nota. \\[2ex]

$\displaystyle \frac{10}{100} \cdot \left( \frac{30}{100} \cdot N \right)$ &
Ahora calculamos el 10\% de ese resultado. \\[2ex]

$\displaystyle \frac{10}{100} \cdot \frac{30}{100} \cdot N$ &
Multiplicamos las fracciones. \\[2ex]

$\displaystyle \frac{300}{10000} \cdot N$ &
Simplificamos la multiplicación. \\[2ex]

$\displaystyle \frac{3}{100} \cdot N$ &
Reducimos la fracción a su mínima expresión. \\[2ex]

$\displaystyle 0.03 \cdot N$ &
Es equivalente al 3\% de la nota total.\\[2ex]
\end{tabularx}
\endgroup

%__________________________________taller 3__________________________________%
\section{Taller 3}
\subsection{Rectángulos de área fija $24$: determinación de perímetro máximo y mínimo}

Con el objetivo de determinar los valores máximo y mínimo del perímetro 
de todos los rectángulos cuya área es fija e igual a \(24\) unidades cuadradas, 
se define una función que describa la relación entre las dimensiones y su perímetro. 
Dado que las longitudes de los lados deben ser positivas, 
se restringe el dominio a valores positivos. 
Posteriormente, se calculan los puntos críticos y se analiza la concavidad 
de la función para clasificar dichos puntos.

\subsubsection{Definición de la función}

Sea un rectángulo de lados \(x>0\) y \(y>0\) con área fija:
\[
xy=24 \quad\Rightarrow\quad y=\frac{24}{x}.
\]
Su perímetro está dado por:
\[
P(x)=2\bigl(x+y\bigr)=2\!\left(x+\frac{24}{x}\right), \qquad x>0.
\]

\subsubsection{Cálculo: mínimo y ausencia de máximo}

Derivando la función \(P\) dos veces con respecto a \(x\), se obtiene:
\[
P'(x)=2\!\left(1-\frac{24}{x^2}\right),\qquad
P''(x)=\frac{96}{x^3}>0.
\]
La ecuación crítica se obtiene al resolver \(P'(x) = 0\):
\[
x^2 = 24 \quad \Rightarrow \quad x = \sqrt{24}.
\]

Como el dominio está restringido a \(x>0\), 
la segunda derivada \(P''(x)\) es estrictamente positiva en todo el dominio, 
lo que indica que la función es cóncava hacia arriba. 
En consecuencia, el punto crítico corresponde a un mínimo global. 
Además, dado que la concavidad no cambia de signo, 
la función no presenta máximos en su dominio.
Por lo tanto, el rectángulo de menor perímetro (manteniendo el área fija) 
es un \textbf{cuadrado}. 
Por otro lado, se tiene que:
\[
\lim_{x\to 0^+} P(x)=\lim_{x\to\infty} P(x)=\infty,
\]
lo que muestra que \textbf{no existe un perímetro máximo}, 
ya que este puede crecer indefinidamente al hacer uno de los lados muy pequeño 
y el otro muy grande.

\subsubsection{Diez rectángulos distintos de área 24}
\[
(x,y)=(x,\tfrac{24}{x}),\qquad
P=2\!\left(x+\frac{24}{x}\right).
\]

\begin{center}
\begingroup
\sisetup{round-mode=places,round-precision=1}
\begin{tabular}{@{}S[table-format=2.1] S[table-format=2.0] S[table-format=3.1]@{}}
\toprule
{$x$} & {$y=\frac{24}{x}$} & {$P$} \\
\midrule
0.5 & 48  & 97.0 \\
1.0 & 24  & 50.0 \\
1.5 & 16  & 35.0 \\
2.0 & 12  & 28.0 \\
3.0 & 8   & 22.0 \\
4.0 & 6   & 20.0 \\
6.0 & 4   & 20.0 \\
8.0 & 3   & 22.0 \\
12.0 & 2  & 28.0 \\
24.0 & 1  & 50.0 \\
\bottomrule
\end{tabular}
\endgroup
\end{center}

Obsérvese cómo el perímetro se acerca al mínimo alrededor de $x=\sqrt{24}\approx \num{4.899}$
y crece sin cota cuando $x\to 0^+$ o $x\to \infty$.

% Galería de 10 rectángulos como figura continuada. Añadimos [tp] para que priorice top o página de flotantes.
\FloatBarrier
\begin{figure}[htbp]
\centering
\setlength{\tabcolsep}{6pt}
\renewcommand{\arraystretch}{1}
\begin{tabular}{@{}cc@{}}
\subcaptionbox{0.5 × 48\label{fig:r1}}{\includegraphics[width=0.4\textwidth]{rect_alto_0p5x48.pdf}} &
\subcaptionbox{1 × 24\label{fig:r2}}{\includegraphics[width=0.4\textwidth]{rect_alto_1x24.pdf}} \\
\subcaptionbox{1.5 × 16\label{fig:r3}}{\includegraphics[width=0.4\textwidth]{rect_alto_1p5x16.pdf}} &
\subcaptionbox{2 × 12\label{fig:r4}}{\includegraphics[width=0.4\textwidth]{rect_alto_2x12.pdf}} \\
\subcaptionbox{3 × 8\label{fig:r5}}{\includegraphics[width=0.4\textwidth]{rect_alto_3x8.pdf}} &
\subcaptionbox{4 × 6\label{fig:r6}}{\includegraphics[width=0.4\textwidth]{rect_alto_4x6.pdf}} \\
\subcaptionbox{6 × 4\label{fig:r7}}{\includegraphics[width=0.4\textwidth]{rect_ancho_6x4.pdf}} &
\subcaptionbox{8 × 3\label{fig:r8}}{\includegraphics[width=0.4\textwidth]{rect_ancho_8x3.pdf}} \\
\subcaptionbox{12 × 2\label{fig:r9}}{\includegraphics[width=0.4\textwidth]{rect_ancho_12x2.pdf}} &
\subcaptionbox{24 × 1\label{fig:r10}}{\includegraphics[width=0.4\textwidth]{rect_ancho_24x1.pdf}} \\
\end{tabular}
\caption{Diez rectángulos distintos de área 24.}
\label{fig:rects_area24}
\end{figure}
\FloatBarrier
  
\subsection{Círculo de área fija 24}
En la Figura~\ref{fig:circle24} se muestra un círculo cuya área es de $24\ \mathrm{u}^2$. 
Se incluye la fórmula general del área de un círculo y el valor del radio calculado 
a partir de dicha área:

\[
A = \pi r^2
\quad\Rightarrow\quad
r = \sqrt{\frac{A}{\pi}} \approx 2.764\ \mathrm{u}.
\]

\begin{figure}[htbp]
    \centering
    \includegraphics[width=0.5\textwidth]{circle_area_24_labeled.pdf}
    \caption{Círculo con $A = 24\ \mathrm{u}^2$, radio calculado y fórmula usada.}
    \label{fig:circle24}
\end{figure}
%__________________________________taller 4__________________________________%
\section{Taller 4: Demostraciones de potenciación + análisis de IA}

\subsection{Propiedades de las potencias para $m,n \in \mathbb{N}$}

\subsubsection{Definición de potencia}

Trabajaremos con $a \in \mathbb{R}$ y $m,n \in \mathbb{N}$.  
Definimos la potencia entera no negativa por producto iterado:
\[
a^n \;=\; \underbrace{a \cdot a \cdot \dotsb \cdot a}_{n \text{ veces}}
\quad (n \in \mathbb{N}),
\]

\medskip
\noindent\textbf{Propiedad 1.} \emph{Para todo $m,n \in \mathbb{N}$ se cumple}
\[
a^m \cdot a^n \;=\; a^{m+n}.
\]

\subsubsection{Demostración por definición de: \(a^m \cdot a^n \;=\; a^{m+n}\)}

\[
a^m \;=\; \underbrace{a \cdot a \cdot \dotsb \cdot a}_{m \text{ veces}},
\qquad
a^n \;=\; \underbrace{a \cdot a \cdot \dotsb \cdot a}_{n \text{ veces}}.
\]
Al multiplicar $a^m \cdot a^n$,
primero \textbf{usamos la propiedad conmutativa} para observar que todos los factores son iguales y podemos reordenarlos sin alterar el producto.  
Luego, \textbf{usamos la propiedad asociativa} para eliminar paréntesis y escribir la multiplicación como una sola secuencia continua de factores:
\[
a^m \cdot a^n
= 
(\underbrace{a \cdot a \cdots a}_{m \text{ veces}})
\cdot 
(\underbrace{a \cdot a \cdots a}_{n \text{ veces}})
=
\underbrace{a \cdot a \cdots a}_{m+n \text{ veces}}
= a^{m+n}.
\]
Así, la concatenación de los $m$ factores de $a^m$ y los $n$ factores de $a^n$
produce $m+n$ factores en total, que por definición es $a^{m+n}$.
\qedhere

\medskip
\noindent\textbf{Propiedad 2.} \emph{Para todo $m,n \in \mathbb{N}$ se cumple}
\[
\bigl(a^m\bigr)^n \;=\; a^{mn}.
\]

\subsubsection{Demostración por definición de: \((a^m)^n \;=\; a^{mn}\)}

\[
\bigl(a^m\bigr)^n
\;=\;
\underbrace{a^m \cdot a^m \cdot \dotsb \cdot a^m}_{n \text{ veces}}.
\]
Cada factor $a^m$ es, a su vez,
\[
a^m = \underbrace{a \cdot a \cdots a}_{m \text{ veces}}.
\]
Entonces, \textbf{usando la propiedad conmutativa} podemos considerar que todos los factores $a$ de todos los $a^m$ son indistinguibles y pueden colocarse en cualquier orden.  
Después, \textbf{usando la propiedad asociativa} podemos reagrupar todos estos factores en una sola multiplicación continua:
\[
\bigl(a^m\bigr)^n
= 
\underbrace{
\left( \underbrace{a \cdot a \cdots a}_{m \text{ veces}} \right)
\cdot
\left( \underbrace{a \cdot a \cdots a}_{m \text{ veces}} \right)
\cdots
}_{n \text{ veces}}
= 
\underbrace{a \cdot a \cdots a}_{m \cdot n \text{ veces}}.
\]
Por definición de potencia, esto último es $a^{mn}$.

\qedhere

\medskip
\subsubsection{Convención para exponente cero.}
Definimos $a^0:=1$ para todo $a\neq 0$, donde $1$ es el elemento neutro multiplicativo.  
Con esta elección, para todo $m\in\mathbb{N}$ se mantiene la propiedad:
\[
a^m \cdot a^0 = a^{m+0} = a^m.
\]
Esta igualdad solo es posible si $a^0 = 1$ para todo $a \neq 0$.  
(En este contexto, $0^0$ no se define.)

\subsection{Interacción con IA}

\subsubsection{Demostración de propiedades de potencias}

\textbf{Enunciado(prompt):}  
Piensa tu respuesta y a partir de las definiciones pruebe que:
\[
a^m \cdot a^n = a^{m+n} \quad \text{para } m,n \in \mathbb{N}
\]
\[
(a^m)^n = a^{mn} \quad \text{para } m,n \in \mathbb{N}
\]
\[
a^0 = 1
\]

\bigskip

\textbf{Demostración de la IA:}  

Trabajaremos en un conjunto con una operación multiplicativa asociativa y elemento neutro \(1\) (un monoide).  
Para \(a\) fijo, definimos potencias naturales de forma recursiva:
\[
a^1 := a, \quad a^{n+1} := a^n \cdot a \quad \text{para } n \in \mathbb{N}.
\]

\subsection*{1) \(a^m \cdot a^n = a^{m+n}\) para \(m,n \in \mathbb{N}\)}

\textbf{Demostración por inducción en \(n\):}

\textbf{Caso base:} Para \(n=1\):
\[
a^m \cdot a = a^{m+1}
\]
por la definición recursiva. Es decir, \(a^m \cdot a^1 = a^{m+1}\).

\textbf{Paso inductivo:} Supongamos que para algún \(n\) se cumple \(a^m \cdot a^n = a^{m+n}\).  
Entonces:
\[
\begin{aligned}
a^m \cdot a^{n+1} 
&= a^m \cdot (a^n \cdot a) \\
&= (a^m \cdot a^n) \cdot a \quad \text{(asociatividad)} \\
&= a^{m+n} \cdot a \\
&= a^{m+n+1} \\
&= a^{m+(n+1)}.
\end{aligned}
\]
Por inducción, queda probado para todo \(n\).

\subsection*{2) \((a^m)^n = a^{mn}\) para \(m,n \in \mathbb{N}\)}

\textbf{Demostración por inducción en \(n\):}

\textbf{Caso base:} Para \(n=1\):
\[
(a^m)^1 = a^m = a^{m \cdot 1}.
\]

\textbf{Paso inductivo:} Supongamos que \((a^m)^n = a^{mn}\). Entonces:
\[
\begin{aligned}
(a^m)^{n+1} 
&= (a^m)^n \cdot a^m \\
&= a^{mn} \cdot a^m \\
&= a^{mn+m} \\
&= a^{m(n+1)},
\end{aligned}
\]
donde usamos el resultado del punto 1). Por inducción, la igualdad es válida para todo \(n\).

\subsection*{3) \(a^0 = 1\)}

Para incorporar el exponente \(0\) de forma coherente con el punto (1), la única elección posible es \(a^0 = 1\).

Si en (1) ponemos \(n=0\):
\[
a^m \cdot a^0 = a^{m+0} = a^m \quad \text{para todo } m.
\]
Esto significa que \(a^0\) actúa como elemento neutro derecho para todos los \(a^m\).  
El elemento neutro de la multiplicación es único y es \(1\), por lo tanto:
\[
a^0 = 1.
\]

\subsubsection{Explicación con mis palabras:}
La IA realiza la demostración de las propiedades propuestas a partir de 
la definición recursiva de las potencias \(a^1 := a\) y \(a^{n+1} := a^n \cdot a\).  
Utiliza el método de inducción matemática para probar las dos primeras propiedades: 
la suma de exponentes y la multiplicación de exponentes.  
En ambos casos, se parte del \textbf{caso base} y luego se realiza el \textbf{paso inductivo}, 
usando principalmente la definición \(a^{n+1} = a^n \cdot a\) para mostrar que, asumiendo que 
la propiedad es cierta para un número \(n\), también lo es para \(n+1\).  
Finalmente, para el caso \(a^0 = 1\), se justifica que debe ser el elemento neutro multiplicativo 
para que las leyes de potencias se mantengan.


\end{document}
