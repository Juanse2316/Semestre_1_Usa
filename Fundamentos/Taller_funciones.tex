\documentclass[12pt,letterpaper]{exam}

\usepackage[left=2cm,top=2cm,right=2cm,bottom=2.5cm]{geometry}
\usepackage{hyperref}
\usepackage[T1]{fontenc}
\usepackage[utf8]{inputenc}
\DeclareUnicodeCharacter{001C}{}
\usepackage[spanish,activeacute]{babel}
\decimalpoint

\usepackage{enumerate}
\usepackage{eurosym}
\usepackage{latexsym,amsmath,amsthm,amssymb,amsfonts,bbm,dsfont}
\usepackage{mathrsfs}

% --- TikZ ---
\usepackage{tikz}
\usetikzlibrary{calc,arrows.meta,intersections,positioning}
\tikzset{
  plane/.style={fill=gray!12, draw=black!50},
  line3d/.style={line width=0.9pt},
  pt/.style={circle,fill=black,inner sep=1.4pt},
  label/.style={font=\small}
}

% --- Imágenes y colores ---
\usepackage{graphicx}
\DeclareGraphicsExtensions{.pdf,.png,.jpg}
\usepackage{adjustbox}
\usepackage[dvipsnames]{xcolor}

% --- Caption / subfiguras ---
\usepackage[font=small,labelfont=bf,textfont=it]{caption}
\usepackage{subcaption}

\usepackage{enumitem}
\usepackage{float}
\usepackage{multirow}
\usepackage{fancybox}

% === RUTAS DE APOYO (opcional) ===
\graphicspath{{../assets/}{../Talleres_fundamentos/}{./}}

%--------------------------------------------------------------------
\newcommand{\base}[1]{\underline{\hspace{#1}}}
%--------------------------------------------------------------------
\newcommand{\uni}{Universidad Sergio Arboleda}
\newcommand{\fac}{\normalsize{Escuela de Ciencias Ex\'actas e Ingenier\'ia}}
\newcommand{\dep}{Matem\'aticas}
\newcommand{\mat}{Materia} %Materia
\newcommand{\tema}{} %Tipo y Número de Quiz
\newcommand{\autor}{Profesor}% nombre del profesor 
\newcommand{\espacio}[1]{\vspace{#1}}
\renewcommand{\arraystretch}{1.15}

% ---- utilidades para conjuntos ----
\newcommand{\CA}[1]{A\setminus #1}
\newcommand{\CB}[1]{B\setminus #1}

%---------------------------------------------------------------------
\pagestyle{headandfoot}
\footrule
\headrule
\firstpageheader{}{}{}
\firstpagefooter{}{\thepage $\,$ de \numpages}{}
\runningfooter{\uni}{\thepage $\,$ de \numpages}{}

\begin{document}

% ===== Encabezado institucional =====
\begin{tabular}{lr}
    \multirow{2}{*}{\includegraphics[height=1.4cm]{logosergio.png}} &
    {\textbf{\uni}} \\
    & {\textbf{\fac}} \\
    & {\textbf{\dep}} \\
    & {\textbf{\mat \tema}} \\
    & {\textit{\autor}} \\
    & {\textit{}}
\end{tabular}\\
\base{19.5cm}\\
\textbf{Nombre}: \makebox[11.2cm]{\hfill Juan Ladino Juan Casas \hfill} \quad 
\textit{Calificaci\'on}: \base{2cm} \\[6pt]

% ===== TALLER 1 =====
\section*{Taller 6 (Funciones)}

% -------------------------------------------------------------------
% LISTA PRINCIPAL DE EJERCICIOS (en lugar de \subsection)
% -------------------------------------------------------------------
\begin{enumerate}[label=\textbf{\arabic*.}, leftmargin=*, itemsep=1.0em]
\item ¿Puede usted, observando la gráfica de una relación, decir si es una función o no? 
        ¿Y mirando el diagrama sagital? ¿Qué criterios emplea?

\item Dibuje el gráfico de una función constante de $\mathbb{R}$ en $\mathbb{R}$.

\item ¿Son las funciones $\sin, \cos : \mathbb{R}\to\mathbb{R}$ inyectivas? ¿Por qué?

\item Halle criterios para saber, observando sólo la gráfica o sólo el diagrama sagital de una función de $A$ en $B$,
    \begin{enumerate}[label=\alph*)]
        \item si es inyectiva, y
        \item si es sobreyectiva.
    \end{enumerate}
\item ¿Es $\varnothing \subseteq A\times B$ una función? 
      ¿Es $\varnothing$ una función de $A$ en $B$? ¿Por qué?

\item \textbf{Im\'agenes de subconjuntos bajo una funci\'on.}
Sean $f:A\to B$ una funci\'on y $M,M_1,M_2\subseteq A$. Por definici\'on
\[
f(M)=\{\,f(x):x\in M\,\},\qquad \CA{M}=A\setminus M,\qquad \CB{S}=B\setminus S.
\]
Pruebe:

\begin{enumerate}[label=\textbf{\alph*)}, topsep=4pt,itemsep=8pt]

% ------------------- (a) -------------------
\item \textbf{(Monoton\'ia)} Si $M_1\subseteq M_2$, entonces $f(M_1)\subseteq f(M_2)$.

\begin{proof}\leavevmode
\[
\begin{aligned}
y\in f(M_1)
&\Leftrightarrow (\exists x\in A)\,[x\in M_1\land f(x)=y] \\
&\Rightarrow (\exists x\in A)\,[x\in M_2\land f(x)=y] \quad(M_1\subseteq M_2)\\
&\Leftrightarrow y\in f(M_2).
\end{aligned}
\]
\end{proof}

% ------------------- (b) -------------------
\item \textbf{(Uni\'on finita)} $f(M_1\cup M_2)=f(M_1)\cup f(M_2)$.

\begin{proof}\leavevmode
\[
\begin{aligned}
y\in f(M_1\cup M_2)
&\Leftrightarrow (\exists x)\,[x\in M_1\cup M_2 \land f(x)=y]\\
&\Leftrightarrow (\exists x)\,[(x\in M_1\lor x\in M_2)\land f(x)=y]\\
&\Leftrightarrow \big((\exists x)[x\in M_1\land f(x)=y]\big)\lor
\big((\exists x)[x\in M_2\land f(x)=y]\big)\\
&\Leftrightarrow y\in f(M_1)\cup f(M_2).
\end{aligned}
\]
La inclusi\'on rec\'iproca es an\'aloga.
\end{proof}

% ------------------- (c) -------------------
\item \textbf{(Uni\'on arbitraria)}
Para toda colecci\'on no vac\'ia $\mathcal M$ de subconjuntos de $A$,
\[
f\!\Big(\bigcup_{M\in\mathcal M}M\Big)\;=\;\bigcup_{M\in\mathcal M} f(M).
\]

\begin{proof}\leavevmode
\[
\begin{aligned}
y\in f\!\Big(\bigcup_{M\in\mathcal M}M\Big)
&\Leftrightarrow (\exists x)\,[x\in\bigcup_{M\in\mathcal M}M\land f(x)=y]\\
&\Leftrightarrow (\exists M\in\mathcal M)(\exists x)\,[x\in M\land f(x)=y]\\
&\Leftrightarrow (\exists M\in\mathcal M)\,[y\in f(M)]
\Leftrightarrow y\in \bigcup_{M\in\mathcal M} f(M).
\end{aligned}
\]
\end{proof}

% ------------------- (d) -------------------
\item \textbf{(Caracterizaci\'on por definici\'on; inyectividad)} 
\[
(\forall M\subseteq A)(\forall x\in A)\ \big(f(x)\in f(M)\ \Leftrightarrow\ x\in M\big)
\quad\text{ssi $f$ es inyectiva.}
\]

\begin{proof}
($\Rightarrow$) Sea $x,y\in A$ con $f(x)=f(y)$. Tomando $M=\{x\}$,
\[
f(y)\in f(\{x\})\ \Rightarrow\ y\in\{x\}\ \Rightarrow\ y=x.
\]
Por definici\'on, $f$ es inyectiva.

($\Leftarrow$) Si $f(x)\in f(M)$, por definici\'on existe $m\in M$ con $f(m)=f(x)$. Si $f$ es inyectiva, $x=m\in M$. El rec\'iproco es inmediato.
\end{proof}

% ------------------- (e) -------------------
\item \textbf{(Complemento e inyectividad)}
\[
(\forall M\subseteq A)\quad f(\CA{M})\subseteq \CB{f(M)}
\quad\text{ssi $f$ es inyectiva.}
\]

\begin{proof}
($\Leftarrow$) Sea $y\in f(\CA{M})$. Entonces $(\exists x)[x\notin M\land f(x)=y]$. 
Si $y\in f(M)$, existir\'ia $m\in M$ con $f(m)=y=f(x)$; por inyectividad $m=x$, imposible. Luego $y\notin f(M)$, i.e. $y\in \CB{f(M)}$.

($\Rightarrow$) Tome $M=\{x\}$. Si $f(y)=f(x)$ con $y\neq x$, entonces $y\in \CA{\{x\}}$ y $f(y)\in f(\CA{\{x\}})\subseteq \CB{f(\{x\})}$, contradicci\'on. Por tanto $y=x$ y $f$ es inyectiva.
\end{proof}

% ------------------- (f) -------------------
\item \textbf{(Complemento y sobreyectividad)}
\[
(\forall M\subseteq A)\quad \CB{f(M)}\subseteq f(\CA{M})
\quad\text{ssi $f$ es sobreyectiva.}
\]

\begin{proof}
($\Leftarrow$) Sea $b\in \CB{f(M)}$. Por sobreyectividad, existe $a\in A$ con $f(a)=b$. Si $a\in M$ entonces $b\in f(M)$, contradicci\'on. Luego $a\in \CA{M}$ y $b=f(a)\in f(\CA{M})$.

($\Rightarrow$) Con $M=\varnothing$ se obtiene $B=\CB{f(\varnothing)}\subseteq f(A)$, esto es, $f$ es sobreyectiva.
\end{proof}

% ------------------- (g) -------------------
\item \textbf{(Diferencia e inyectividad)}
\[
(\forall M_1,M_2\subseteq A)\quad f(M_1\setminus M_2)=f(M_1)\setminus f(M_2)
\quad\text{ssi $f$ es inyectiva.}
\]

\begin{proof}
Para toda $f$ siempre vale
\[
\begin{aligned}
y\in f(M_1\setminus M_2)
&\Leftrightarrow (\exists x)[x\in M_1\land x\notin M_2\land f(x)=y]\\
&\Rightarrow y\in f(M_1)\ \land\ \neg(\exists m\in M_2)\,[f(m)=y]\\
&\Leftrightarrow y\in f(M_1)\setminus f(M_2).
\end{aligned}
\]
Si $f$ es inyectiva y $y\in f(M_1)\setminus f(M_2)$, existe $x\in M_1$ con $f(x)=y$ y no existe $m\in M_2$ con $f(m)=y$; por inyectividad $x\notin M_2$, as\'i $x\in M_1\setminus M_2$ y $y\in f(M_1\setminus M_2)$.

Rec\'iprocamente, si la igualdad vale para todos $M_1,M_2$, con $M_1=\{x\}$ y $M_2=\{y\}$ tendr\'iamos
\[
f(\{x\}\setminus\{y\})=f(\{x\})\setminus f(\{y\}).
\]
Si $x\neq y$ y $f(x)=f(y)$, el lado derecho es $\varnothing$ mientras que el izquierdo es $f(\{x\})\neq\varnothing$, contradicci\'on. Luego $f$ es inyectiva.
\end{proof}

% ------------------- (h) -------------------
\item \textbf{(Bicondici\'on mon\'otona e inyectividad)}
\[
(\forall M_1,M_2\subseteq A)\quad
\big[M_1\subseteq M_2\ \Leftrightarrow\ f(M_1)\subseteq f(M_2)\big]
\quad\text{ssi $f$ es inyectiva.}
\]

\begin{proof}
($\Leftarrow$) La implicaci\'on $M_1\subseteq M_2\Rightarrow f(M_1)\subseteq f(M_2)$ es el punto (a). 
Para la inversa, si $f$ es inyectiva y $f(M_1)\subseteq f(M_2)$, entonces para todo $x\in M_1$ se tiene $f(x)\in f(M_2)$; existe $m\in M_2$ con $f(m)=f(x)$ y por inyectividad $x=m\in M_2$. Por tanto $M_1\subseteq M_2$.

($\Rightarrow$) Suponiendo la bicondici\'on para todos $M_1,M_2$, con $M_1=\{x\}$ y $M_2=\{y\}$: si $f(x)=f(y)$, entonces $f(\{x\})\subseteq f(\{y\})$ y $f(\{y\})\subseteq f(\{x\})$, de donde $\{x\}\subseteq\{y\}$ y $\{y\}\subseteq\{x\}$; as\'i $x=y$ e $f$ es inyectiva.
\end{proof}

\end{enumerate}

% === Propiedades de la preimagen y caracterizaciones ===
\item Sean $f:A\to B$ una función, $N,N_1,N_2\subseteq B$ y $M\subseteq A$.
Escribimos $C_A(S)=A\setminus S$ y $C_B(T)=B\setminus T$. Demuestre:

\begin{enumerate}[label=\textbf{\alph*)},itemsep=8pt]

\item \textbf{Monotonía de la preimagen:}
Si $N_1\subseteq N_2$, entonces $f^{-1}(N_1)\subseteq f^{-1}(N_2)$.

\emph{Prueba.}
$x\in f^{-1}(N_1)\Leftrightarrow f(x)\in N_1\Rightarrow f(x)\in N_2\Leftrightarrow x\in f^{-1}(N_2)$.\qed

\item \textbf{Unión finita:}
$f^{-1}(N_1\cup N_2)=f^{-1}(N_1)\cup f^{-1}(N_2)$.

\emph{Prueba.}
\[
\begin{aligned}
x\in f^{-1}(N_1\cup N_2)
&\Leftrightarrow f(x)\in N_1\cup N_2\\
&\Leftrightarrow f(x)\in N_1\ \vee\ f(x)\in N_2\\
&\Leftrightarrow x\in f^{-1}(N_1)\ \vee\ x\in f^{-1}(N_2).
\end{aligned}
\]\qed

\item \textbf{Unión arbitraria:}
Para toda colección $\mathcal N$ de subconjuntos de $B$,
\[
f^{-1}\!\Big(\bigcup_{N\in\mathcal N}N\Big)=\bigcup_{N\in\mathcal N} f^{-1}(N).
\]

\emph{Prueba.}
$x\in f^{-1}(\bigcup_{N\in\mathcal N}N)\Leftrightarrow f(x)\in\bigcup_{N\in\mathcal N}N\Leftrightarrow
(\exists N\in\mathcal N)\, f(x)\in N \Leftrightarrow (\exists N\in\mathcal N)\, x\in f^{-1}(N)$.\qed

\item \textbf{Intersección finita:}
$f^{-1}(N_1\cap N_2)=f^{-1}(N_1)\cap f^{-1}(N_2)$.

\emph{Prueba.}
\[
\begin{aligned}
x\in f^{-1}(N_1\cap N_2)
&\Leftrightarrow f(x)\in N_1\cap N_2\\
&\Leftrightarrow f(x)\in N_1\ \wedge\ f(x)\in N_2\\
&\Leftrightarrow x\in f^{-1}(N_1)\ \wedge\ x\in f^{-1}(N_2).
\end{aligned}
\]\qed

\item \textbf{Intersección arbitraria:}
Para toda colección $\mathcal N$ de subconjuntos de $B$,
\[
f^{-1}\!\Big(\bigcap_{N\in\mathcal N}N\Big)=\bigcap_{N\in\mathcal N} f^{-1}(N).
\]

\emph{Prueba.}
$x\in f^{-1}(\bigcap_{N\in\mathcal N}N)\Leftrightarrow f(x)\in\bigcap_{N\in\mathcal N}N
\Leftrightarrow(\forall N\in\mathcal N)\, f(x)\in N\Leftrightarrow (\forall N\in\mathcal N)\, x\in f^{-1}(N)$.\qed

\item \textbf{Diferencia:}
$f^{-1}(N_1-N_2)=f^{-1}(N_1)-f^{-1}(N_2)$.

\emph{Prueba.}
\[
\begin{aligned}
x\in f^{-1}(N_1-N_2)
&\Leftrightarrow f(x)\in N_1\ \wedge\ f(x)\notin N_2\\
&\Leftrightarrow x\in f^{-1}(N_1)\ \wedge\ x\notin f^{-1}(N_2)\\
&\Leftrightarrow x\in f^{-1}(N_1)-f^{-1}(N_2).
\end{aligned}
\]\qed

\item \textbf{Complemento:}
$f^{-1}(C_B N)=C_A\big(f^{-1}(N)\big)$.

\emph{Prueba.}
$x\in f^{-1}(C_B N)\Leftrightarrow f(x)\notin N\Leftrightarrow x\notin f^{-1}(N)\Leftrightarrow x\in C_A(f^{-1}(N))$.\qed

\item \textbf{Imagen de la preimagen (inclusión):}
$f\!\big(f^{-1}(N)\big)\subseteq N$.

\emph{Prueba.}
Si $y\in f(f^{-1}(N))$, existe $x$ con $x\in f^{-1}(N)$ y $f(x)=y$; pero $x\in f^{-1}(N)\Leftrightarrow f(x)\in N$, así $y\in N$.\qed

\item \textbf{Caracterización de sobreyectividad:}
\[
(\forall N\subseteq B)\quad f(f^{-1}(N))=N
\quad\text{ssi $f$ es sobreyectiva.}
\]

\emph{Prueba.}
($\Rightarrow$) Con $N=B$, se obtiene $f(A)=B$.  
($\Leftarrow$) La inclusión $\subseteq$ es (h). Para la opuesta, si $f$ es sobreyectiva y $y\in N$, elija $x\in A$ con $f(x)=y$; entonces $x\in f^{-1}(N)$ y $y=f(x)\in f(f^{-1}(N))$.\qed

\item \textbf{Subconjunto “siempre”:}
$M\subseteq f^{-1}(f(M))$.

\emph{Prueba.}
Si $x\in M$, entonces $f(x)\in f(M)$; por definición, $x\in f^{-1}(f(M))$.\qed

\item \textbf{Caracterización de inyectividad:}
\[
(\forall M\subseteq A)\quad M=f^{-1}(f(M))
\quad\text{ssi $f$ es inyectiva.}
\]

\emph{Prueba.}
($\Leftarrow$) Para toda $f$ ya sabemos $M\subseteq f^{-1}(f(M))$ (j). Si $f$ es inyectiva y $x\in f^{-1}(f(M))$, existe $m\in M$ con $f(x)=f(m)$; por inyectividad $x=m\in M$.  
($\Rightarrow$) Tomando $M=\{x\}$ se tiene $\{x\}=f^{-1}(f(\{x\}))$. Si $f(x)=f(y)$, entonces $y\in f^{-1}(f(\{x\}))=\{x\}$, luego $y=x$; así $f$ es inyectiva.\qed

\item \textbf{Orden reflejado por la preimagen y sobreyectividad:}
\[
(\forall N_1,N_2\subseteq B)\quad
\big[N_1\subseteq N_2\ \Leftrightarrow\ f^{-1}(N_1)\subseteq f^{-1}(N_2)\big]
\quad\text{ssi $f$ es sobreyectiva.}
\]

\emph{Prueba.}
($\Leftarrow$) Si $f$ es sobreyectiva y $f^{-1}(N_1)\subseteq f^{-1}(N_2)$, sea $y\in N_1$. Elija $x\in A$ con $f(x)=y$. Entonces $x\in f^{-1}(N_1)$ y por la hipótesis $x\in f^{-1}(N_2)$, esto es $f(x)=y\in N_2$. Concluye $N_1\subseteq N_2$.  
La implicación $N_1\subseteq N_2\Rightarrow f^{-1}(N_1)\subseteq f^{-1}(N_2)$ es el punto (a).

($\Rightarrow$) Suponga la bicondición. Para $S\subseteq B$ se tiene siempre $f^{-1}(S)\subseteq f^{-1}(f(f^{-1}(S)))$; por hipótesis, $S\subseteq f(f^{-1}(S))$. Junto con (h) se deduce $f(f^{-1}(S))=S$ para todo $S$, y por (i) $f$ es sobreyectiva.\qed

\item \textbf{Intersección con preimagen:}
\[
f\big(M\cap f^{-1}(N)\big)=f(M)\cap N.
\]

\emph{Prueba.}
$\subseteq$: si $y\in f(M\cap f^{-1}(N))$, existe $x\in M\cap f^{-1}(N)$ con $f(x)=y$; entonces $y\in f(M)$ y, como $x\in f^{-1}(N)$, también $y=f(x)\in N$.  
$\supseteq$: si $y\in f(M)\cap N$, existe $x\in M$ con $f(x)=y$ y, como $y\in N$, se tiene $x\in f^{-1}(N)$. Luego $x\in M\cap f^{-1}(N)$ y $y\in f(M\cap f^{-1}(N))$.\qed

\end{enumerate}


%__________________8
\item Sea $f:A\to B$; definimos otra función
\[
\hat f:\mathcal P(A)\to\mathcal P(B), 
\qquad \text{así: si } M\in\mathcal P(A),\quad 
\hat f(M)=\{\,f(x)\mid x\in M\,\}.
\]
Demuestre que:
\begin{enumerate}[label=\alph*)]
  \item $\hat f$ es uno a uno si y sólo si $f$ es uno a uno.
  \item $\hat f$ es sobreyectiva si y sólo si $f$ es sobreyectiva.
\end{enumerate}
\begin{enumerate}[label=\textbf{\alph*)}]
\item \textbf{$\hat f$ es inyectiva $\Leftrightarrow$ $f$ es inyectiva.}

\begin{proof}
($\Rightarrow$) Suponga que $\hat f$ es inyectiva. Si $f(x)=f(y)$, entonces
\[
\hat f(\{x\})=\{f(x)\}=\{f(y)\}=\hat f(\{y\}).
\]
Por la inyectividad de $\hat f$, $\{x\}=\{y\}$ y por tanto $x=y$. Así, $f$ es inyectiva.

($\Leftarrow$) Suponga que $f$ es inyectiva y que $\hat f(M_1)=\hat f(M_2)$.
Sea $x\in M_1$. Entonces $f(x)\in \hat f(M_1)=\hat f(M_2)$, de modo que
existe $y\in M_2$ con $f(y)=f(x)$. Por inyectividad de $f$, $y=x$, luego $x\in M_2$.
Así $M_1\subseteq M_2$. Intercambiando los papeles de $M_1,M_2$ se obtiene
$M_2\subseteq M_1$, y por tanto $M_1=M_2$. Luego $\hat f$ es inyectiva.
\end{proof}

\item \textbf{$\hat f$ es sobreyectiva $\Leftrightarrow$ $f$ es sobreyectiva.}

\begin{proof}
($\Rightarrow$) Si $\hat f$ es sobreyectiva, existe $M\subseteq A$ tal que
$\hat f(M)=\mathcal P(B)$ cubre en particular a $B$; basta notar que
\[
B=\hat f(M)\supseteq f(M)\subseteq f(A)\subseteq B,
\]
por lo que $f(A)=B$ y $f$ es sobreyectiva.

(\emph{Versión estándar y más directa}): como $\hat f$ es sobreyectiva, existe $M\subseteq A$ con $\hat f(M)=B$; entonces $B=f(M)\subseteq f(A)\subseteq B$, luego $f(A)=B$.

($\Leftarrow$) Si $f$ es sobreyectiva, entonces para todo $N\subseteq B$ se cumple
\[
\hat f\big(f^{-1}(N)\big)=f\big(f^{-1}(N)\big)=N,
\]
ya que $f(f^{-1}(N))=N$ cuando $f$ es sobreyectiva. Así, dado cualquier
$N\in\mathcal P(B)$, elige $M=f^{-1}(N)\subseteq A$ y se tiene $\hat f(M)=N$.
Por lo tanto, $\hat f$ es sobreyectiva.
\end{proof}

\end{enumerate}

\end{enumerate}


\end{document}
