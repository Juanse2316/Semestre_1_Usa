\documentclass[12pt,letterpaper]{exam}
\usepackage[left=2cm,top=2cm,right=2cm,bottom=2.5cm]{geometry}
\usepackage{hyperref}
\usepackage[utf8]{inputenc}
\usepackage[spanish,activeacute]{babel}
\decimalpoint
\usepackage{enumerate}
\usepackage{eurosym}
\usepackage{latexsym,amsmath,amsthm,amssymb,amsfonts,bbm, dsfont}
\usepackage[mathscr]{euscript}
\usepackage{ae,aecompl}
% --- Imágenes y colores ---
\usepackage{graphicx}
\DeclareGraphicsExtensions{.pdf,.png,.jpg}
\usepackage{adjustbox}
\usepackage[dvipsnames]{xcolor}
%---- modificar las caption -----
\usepackage[font=small,labelfont=small,labelfont=bf,textfont=it]{caption}
\usepackage{enumitem}
\usepackage{float}
\usepackage{subfigure}
\usepackage{multirow}
\usepackage{fancybox}
% === RUTAS DE APOYO (opcional) ===
\graphicspath{{../assets/}{../Talleres_fundamentos/}{./}}

%--------------------------------------------------------------------
\newcommand{\base}[1]{\underline{\hspace{#1}}}
\newcommand{\uni}{Universidad Sergio Arboleda}
\newcommand{\fac}{\normalsize{Escuela de Ciencias Exáctas e Ingeniería}}
\newcommand{\dep}{Matemáticas}
\newcommand{\mat}{Fundamentos de Matem\'aticas}
\newcommand{\tema}{Taller de Conjuntos – Soluciones}
\newcommand{\autor}{Adriana M. Salinas}
\newcommand{\espacio}[1]{\vspace{#1}}

%---------------------------------------------------------------------
\pagestyle{headandfoot}
\footrule
\headrule
\firstpageheader{}{}{}
\firstpagefooter{}{\thepage $\,$ de \numpages}{}
\runningfooter{\uni}{\thepage $\,$ de \numpages}{}

\begin{document}

% ===== Encabezado institucional =====
\begin{tabular}{lr}
    \multirow{2}{*}{\includegraphics[height=1.4cm]{logosergio.png}} &
    {\textbf{\uni}} \\
    & {\textbf{\fac}} \\
    & {\textbf{\dep}} \\
    & {\textbf{\mat \tema}} \\
    & {\textit{\autor}} \\
    & {\textit{}}
\end{tabular}\\
\base{19.5cm}\\
\textbf{Nombre}: \makebox[11.2cm]{\hfill Juan Ladino Juan Casas \hfill} \quad 
\textit{Calificación}: \base{2cm} \\[6pt]

% ======================================================
\begin{questions}

% ---------- PROBLEMA 1 ----------
\question \textbf{(1)} Para cada $n\in\mathbb N$, se da $B_n$. Halle $\displaystyle\bigcup_{n\in\mathbb N}B_n$ y $\displaystyle\bigcap_{n\in\mathbb N}B_n$.
\begin{enumerate}[label=\alph*)]
\item $B_n=\{0,1,2,\ldots,2n\}.$
\[
\bigcup_{n\in\mathbb N}B_n=\{0,1,2,\ldots\},\qquad
\bigcap_{n\in\mathbb N}B_n=\{0,1,2\}.
\]

\item $B_n=\{n-1,n,n+1\}.$
\[
\bigcup_{n\in\mathbb N}B_n=\{0,1,2,\ldots\},\qquad
\bigcap_{n\in\mathbb N}B_n=\varnothing.
\]

\item $B_n=\left[\frac{3}{n},\,\frac{5n+2}{n}\right)\ \cup\ \{\,10+n\,\}.$
\[
\bigcup_{n\in\mathbb N}B_n=(0,7)\ \cup\ \{11,12,13,\ldots\},\qquad
\bigcap_{n\in\mathbb N}B_n=(0,5].
\]

\item $B_n=\left[-1,\,3+\frac{1}{n}\right]\ \cup\ \left[5,\,\frac{5n+1}{n}\right]
=\left[-1,\,3+\frac{1}{n}\right]\ \cup\ \left[5,\,5+\frac{1}{n}\right].$
\[
\bigcup_{n\in\mathbb N}B_n=[-1,4]\ \cup\ [5,6],\qquad
\bigcap_{n\in\mathbb N}B_n=[-1,3]\ \cup\ \{5\}.
\]

\item $B_n=\left(-\frac{1}{n},\,1\right]\ \cup\ \left(2,\,\frac{3n-1}{n}\right]
=\left(-\frac{1}{n},\,1\right]\ \cup\ \left(2,\,3-\frac{1}{n}\right].$
\[
\bigcup_{n\in\mathbb N}B_n=(-1,1]\ \cup\ (2,3),\qquad
\bigcap_{n\in\mathbb N}B_n=[0,1].
\]

\item $B_n=\left[0,\,\frac{n+1}{n+2}\right]\ \cup\ \left[7,\,\frac{7n+1}{n}\right]
=\left[0,\,1-\frac{1}{n+2}\right]\ \cup\ \left[7,\,7+\frac{1}{n}\right].$
\[
\bigcup_{n\in\mathbb N}B_n=[0,1)\ \cup\ [7,8],\qquad
\bigcap_{n\in\mathbb N}B_n=\left[0,\,\frac{2}{3}\right]\ \cup\ \{7\}.
\]
\end{enumerate}

% ---------- PROBLEMA 2 ----------
\question \textbf{(2)} Construya familias $\{E_n\}_{n\in\mathbb N}\subset\mathbb R$ (todos distintos) que cumplan las condiciones dadas.
\begin{enumerate}[label=\alph*)]
\item $\displaystyle\bigcup_{n}E_n=[0,\infty)$ y $\displaystyle\bigcap_{n}E_n=[0,1]$.\\
$E_n=[0,n]\ (n\ge1)$.

\item $\displaystyle\bigcup_{n}E_n=(0,\infty)$ y $\displaystyle\bigcap_{n}E_n=\varnothing$.\\
$E_n=\big(\tfrac1n,\,n\big)\ (n\ge2)$.

\item $\displaystyle\bigcup_{n}E_n=\mathbb R$ y $\displaystyle\bigcap_{n}E_n=\{3\}$.\\
\(
E_n=
\begin{cases}
\mathbb R\setminus(-n,\,3-\tfrac1n) & \text{si $n$ impar},\\
\mathbb R\setminus(3+\tfrac1n,\,n) & \text{si $n$ par}.
\end{cases}
\)

\item $\displaystyle\bigcup_{n}E_n=(2,8)$ y $\displaystyle\bigcap_{n}E_n=[3,6]$.\\
$E_n=(2+\tfrac1n,\,8-\tfrac1n)\cup[3,6]$.

\item $\displaystyle\bigcup_{n}E_n=[0,\infty)$ y $\displaystyle\bigcap_{n}E_n=\{1\}\cup[2,3)$.\\
$E_{2k-1}=[0,\,3+\tfrac1k),\ 
E_{2k}=\{1\}\cup[2,\,2k+2)\ (k\ge2)$.

\item $\displaystyle\bigcup_{n}E_n=\mathbb Z$ y $\displaystyle\bigcap_{n}E_n=\{\,\ldots,-2,0,2,4,\ldots\}$.\\
$E_n=\{\text{pares}\}\ \cup\ \{m\in\mathbb Z:\ |m|\le n\ \text{y $m$ impar}\}$.

\item \textit{No existe} tal familia con todos $E_n$ distintos que satisfaga
\(
\bigcup_{n}E_n=\bigcap_{n}E_n=\mathbb N.
\)
\end{enumerate}

% ---------- PROBLEMA 3 ----------
\question \textbf{(3) Distributivas con familias}

\textbf{(i)} \quad $
M \cap \Big( \displaystyle\bigcup_{A\in\mathcal C} A \Big)
= \displaystyle\bigcup_{A\in\mathcal C} (M \cap A).
$
\medskip

Tome $x$ arbitrario.
\[
\begin{aligned}
x \in M \cap \Big(\bigcup_{A\in\mathcal C} A\Big)
&\Leftrightarrow (x\in M)\land\big(x\in\textstyle\bigcup_{A\in\mathcal C}A\big)\quad(\text{def. }\cap)\\
&\Leftrightarrow (x\in M)\land\exists A\in\mathcal C\,(x\in A)\quad(\text{def. }\cup)\\
&\Leftrightarrow \exists A\in\mathcal C\,\big((x\in M)\land(x\in A)\big)\quad(p\land\exists\ \iff\ \exists(p\land\cdot))\\
&\Leftrightarrow \exists A\in\mathcal C\,(x\in M\cap A)\quad(\text{def. }\cap)\\
&\Leftrightarrow x\in\textstyle\bigcup_{A\in\mathcal C}(M\cap A)\quad(\text{def. }\cup).
\end{aligned}
\]

\textbf{(ii)} \quad $
M \cup \Big( \displaystyle\bigcap_{A\in\mathcal C} A \Big)
= \displaystyle\bigcap_{A\in\mathcal C} (M \cup A).
$
\medskip

Tome $x$ arbitrario.
\[
\begin{aligned}
x\in M\cup\Big(\bigcap_{A\in\mathcal C}A\Big)
&\Leftrightarrow (x\in M)\vee\big(x\in\textstyle\bigcap_{A\in\mathcal C}A\big)\quad(\text{def. }\cup)\\
&\Leftrightarrow (x\in M)\vee\forall A\in\mathcal C\,(x\in A)\quad(\text{def. }\cap)\\
&\Leftrightarrow \neg\big(\neg(x\in M)\land\neg\forall A\,(x\in A)\big)\quad(\text{De Morgan prop.})\\
&\Leftrightarrow \neg\big(\neg(x\in M)\land\exists A\,\neg(x\in A)\big)\quad(\neg\forall\iff\exists\neg)\\
&\Leftrightarrow \neg\,\exists A\,\big(\neg(x\in M)\land\neg(x\in A)\big)\quad(\text{mover }\exists)\\
&\Leftrightarrow \forall A\,\neg\big(\neg(x\in M)\land\neg(x\in A)\big)\quad(\neg\exists\iff\forall\neg)\\
&\Leftrightarrow \forall A\,\big((x\in M)\vee(x\in A)\big)\quad(\text{De Morgan prop.})\\
&\Leftrightarrow x\in\textstyle\bigcap_{A\in\mathcal C}(M\cup A)\quad(\text{def. }\cap).
\end{aligned}
\]
% ---------- PROBLEMA 4 ----------
\question \textbf{(4) De Morgan relativos en $M$ (familias)}

Recordemos: $x\in M-A\ \Leftrightarrow\ (x\in M)\land\neg(x\in A)$.

\textbf{(i)} \quad $
M - \Big( \displaystyle\bigcup_{A\in\mathcal C} A \Big)
= \displaystyle\bigcap_{A\in\mathcal C} (M - A).
$
\medskip

Tome $x$ arbitrario.
\[
\begin{aligned}
x\in M-\Big(\bigcup_{A\in\mathcal C}A\Big)
&\Leftrightarrow (x\in M)\land\neg\big(x\in\textstyle\bigcup_{A\in\mathcal C}A\big)\quad(\text{def. }-)\\
&\Leftrightarrow (x\in M)\land\neg\exists A\in\mathcal C\,(x\in A)\quad(\text{def. }\cup)\\
&\Leftrightarrow (x\in M)\land\forall A\in\mathcal C\,\neg(x\in A)\quad(\neg\exists\iff\forall\neg)\\
&\Leftrightarrow \forall A\in\mathcal C\,\big((x\in M)\land\neg(x\in A)\big)\quad(p\land\forall\iff\forall(p\land\cdot))\\
&\Leftrightarrow \forall A\in\mathcal C\,(x\in M-A)\quad(\text{def. }-)\\
&\Leftrightarrow x\in\textstyle\bigcap_{A\in\mathcal C}(M-A)\quad(\text{def. }\cap).
\end{aligned}
\]

\textbf{(ii)} \quad $
M - \Big( \displaystyle\bigcap_{A\in\mathcal C} A \Big)
= \displaystyle\bigcup_{A\in\mathcal C} (M - A).
$
\medskip

Tome $x$ arbitrario.
\[
\begin{aligned}
x\in M-\Big(\bigcap_{A\in\mathcal C}A\Big)
&\Leftrightarrow (x\in M)\land\neg\big(x\in\textstyle\bigcap_{A\in\mathcal C}A\big)\quad(\text{def. }-)\\
&\Leftrightarrow (x\in M)\land\neg\forall A\in\mathcal C\,(x\in A)\quad(\text{def. }\cap)\\
&\Leftrightarrow (x\in M)\land\exists A\in\mathcal C\,\neg(x\in A)\quad(\neg\forall\iff\exists\neg)\\
&\Leftrightarrow \exists A\in\mathcal C\,\big((x\in M)\land\neg(x\in A)\big)\quad(p\land\exists\iff\exists(p\land\cdot))\\
&\Leftrightarrow \exists A\in\mathcal C\,(x\in M-A)\quad(\text{def. }-)\\
&\Leftrightarrow x\in\textstyle\bigcup_{A\in\mathcal C}(M-A)\quad(\text{def. }\cup).
\end{aligned}
\]

% ---------- PROBLEMA 5 ----------
\question \textbf{(5)} Si $A_n=\Big(-\frac1n,\,3+\frac1n\Big)$, entonces
\[
\bigcap_{n=2}^{\infty}A_n=\,[0,3].
\]

% ---------- PROBLEMA 6 ----------
\question \textbf{(6)} Si $B_n=\Big[-\frac1n,\,1-\frac1n\Big]$, entonces
\[
\bigcup_{n=2}^{\infty}B_n=\Big[-\frac12,\,1\Big).
\]

% ---------- PROBLEMA 7 ----------
\question \textbf{(7)} Si $A_0$ es cualquier conjunto de una colección no vacía $\mathcal C$, entonces 
\[
\bigcap_{A\in\mathcal C} A \subseteq A_0.
\]
En particular, si existe $A_0\in\mathcal C$ tal que $A_0\subseteq M$, entonces 
\[
\bigcap_{A\in\mathcal C} A \subseteq M.
\]

\textit{Prueba.} Tome $x$ arbitrario. Si $x\in\bigcap_{A\in\mathcal C}A$, entonces $x\in A$ para \emph{todo} $A\in\mathcal C$; en particular $x\in A_0$. Por tanto $\bigcap_{A\in\mathcal C}A\subseteq A_0$. Si además $A_0\subseteq M$, por transitividad
\(
\bigcap_{A\in\mathcal C}A\subseteq A_0\subseteq M.
\)
\hfill$\square$

% ---------- PROBLEMA 8 ----------
\question \textbf{(8)} Si $(\forall A\in\mathcal C)(M\subseteq A)$, entonces 
\[
M\subseteq \bigcap_{A\in\mathcal C}A.
\]

\textit{Prueba.} Sea $x\in M$. Como $M\subseteq A$ para todo $A\in\mathcal C$, se tiene $x\in A$ para todo $A\in\mathcal C$; luego $x\in\bigcap_{A\in\mathcal C}A$. Concluye $M\subseteq\bigcap_{A\in\mathcal C}A$.
\hfill$\square$

% ---------- PROBLEMA 9 ----------
\question \textbf{(9)} Si $A_n=\{x\in\mathbb R:x\ge n\}=[n,\infty)$, halle 
\(
\displaystyle\bigcap_{n=2}^{\infty}A_n.
\)

\textit{Respuesta:} \(
\displaystyle\bigcap_{n=2}^{\infty}A_n=\varnothing.
\)

\textit{Razón.} Si $x\in\bigcap_{n\ge2}A_n$, entonces $x\ge n$ para \emph{todo} $n\ge2$. Elija $n>x+1$ (existe porque $\mathbb N$ no está acotado); entonces $x\not\ge n$, contradicción. Por tanto la intersección es vacía.
\hfill$\square$

% ---------- PROBLEMA 10 ----------
\question \textbf{(10)} Sean $\mathcal A$ y $\mathcal B$ familias no vacías de conjuntos con $\mathcal A\subseteq\mathcal B$. Pruebe:

\begin{enumerate}[label=\alph*)]
\item \(\displaystyle \bigcup_{X\in\mathcal A} X \subseteq \bigcup_{Y\in\mathcal B} Y.\)

\textit{Prueba.} Tome $x$ arbitrario.
\[
\begin{aligned}
x\in \bigcup_{X\in\mathcal A}X 
&\Leftrightarrow \exists\,X_0\in\mathcal A\ (x\in X_0)\\
&\Rightarrow \exists\,Y_0\in\mathcal B\ (x\in Y_0)\quad(\text{pues }X_0\in\mathcal B)\\
&\Leftrightarrow x\in \bigcup_{Y\in\mathcal B}Y.
\end{aligned}
\]
Así, $\bigcup\mathcal A\subseteq\bigcup\mathcal B$.
\hfill$\square$

\item \(\displaystyle \bigcap_{Y\in\mathcal B} Y \subseteq \bigcap_{X\in\mathcal A} X.\)

\textit{Prueba.} Tome $x$ arbitrario.
\[
\begin{aligned}
x\in \bigcap_{Y\in\mathcal B}Y
&\Leftrightarrow \forall\,Y\in\mathcal B\ (x\in Y)\\
&\Rightarrow \forall\,X\in\mathcal A\ (x\in X)\quad(\text{pues }\mathcal A\subseteq\mathcal B)\\
&\Leftrightarrow x\in \bigcap_{X\in\mathcal A}X.
\end{aligned}
\]
Luego $\bigcap\mathcal B\subseteq\bigcap\mathcal A$.
\hfill$\square$
\end{enumerate}
\end{questions}


\end{document}
