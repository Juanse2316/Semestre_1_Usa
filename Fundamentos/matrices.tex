\documentclass[12pt,letterpaper]{exam}

\usepackage[left=2cm,top=2cm,right=2cm,bottom=2.5cm]{geometry}
\usepackage{hyperref}
\usepackage[T1]{fontenc}
\usepackage[utf8]{inputenc}
\DeclareUnicodeCharacter{001C}{}
\usepackage[spanish,activeacute]{babel}
\decimalpoint

\usepackage{enumerate}
\usepackage{eurosym}
\usepackage{latexsym,amsmath,amsthm,amssymb,amsfonts,bbm,dsfont}
\usepackage{mathrsfs}

% --- TikZ ---
\usepackage{tikz}
\usetikzlibrary{calc,arrows.meta,intersections,positioning}
\tikzset{
  plane/.style={fill=gray!12, draw=black!50},
  line3d/.style={line width=0.9pt},
  pt/.style={circle,fill=black,inner sep=1.4pt},
  label/.style={font=\small}
}

% --- Imágenes y colores ---
\usepackage{graphicx}
\DeclareGraphicsExtensions{.pdf,.png,.jpg}
\usepackage{adjustbox}
\usepackage[dvipsnames]{xcolor}

% --- Caption / subfiguras ---
\usepackage[font=small,labelfont=bf,textfont=it]{caption}
\usepackage{subcaption}

\usepackage{enumitem}
\usepackage{float}
\usepackage{multirow}
\usepackage{fancybox}

\graphicspath{{../assets/}{../Talleres_fundamentos/}{./}}

%--------------------------------------------------------------------
\newcommand{\base}[1]{\underline{\hspace{#1}}}
%--------------------------------------------------------------------
\newcommand{\uni}{Universidad Sergio Arboleda}
\newcommand{\fac}{\normalsize{Escuela de Ciencias Ex\'actas e Ingenier\'ia}}
\newcommand{\dep}{Matem\'aticas}
\newcommand{\mat}{Introduci\'on al C\'alculo} %Materia
\newcommand{\tema}{} %Tipo y Número de Quiz
\newcommand{\autor}{Javier Gutiérrez}% nombre del profesor 
\newcommand{\espacio}[1]{\vspace{#1}}
\renewcommand{\arraystretch}{1.15}

%---------------------------------------------------------------------
\pagestyle{headandfoot}
\footrule
\headrule
\firstpageheader{}{}{}
\firstpagefooter{}{\thepage $\,$ de \numpages}{}
\runningfooter{\uni}{\thepage $\,$ de \numpages}{}

\begin{document}

% ===== Encabezado institucional =====
\begin{tabular}{lr}
    \multirow{2}{*}{\includegraphics[height=1.2cm]{logosergio.png}} &
    {\textbf{\uni}} \\
    & {\textbf{\fac}} \\
    & {\textbf{\dep}} \\
    & {\textbf{\mat \tema}} \\
    & {\textit{\autor}} \\
    & {\textit{}}
\end{tabular}\\
\base{19.5cm}\\
\textbf{Nombre}: \makebox[11.2cm]{\hfill Juan Sebastian Ladino Mendieta \hfill} \quad 
\textit{Calificaci\'on}: \base{2cm} \\[6pt]

% ===== CONTENIDO =====
\section*{Formas RREF de matrices aumentadas \texorpdfstring{$[A\mid b]$}{[A|b]} por tamaño y caso}

\textbf{Convención.} Los unos son pivotes. Las letras $a,b,c,d,\alpha,\beta,\gamma,\lambda,\mu\in\mathbb{R}$ son parámetros en las entradas no pivote (pueden ser 0).

% ---------------- 2x2 ----------------
\subsection*{2$\times$2 \; (dos rectas en $\mathbb{R}^2$): \; $[A\mid b]\in\mathbb{R}^{2\times 3}$}

\noindent\textbf{(A) Solución única (intersección en un punto), rango $=2$:}
\[
\left[
\begin{array}{cc|c}
1 & 0 & \alpha\\
0 & 1 & \beta
\end{array}
\right]
\]

\noindent\textbf{(B) Inconsistente (paralelas distintas):}
\[
\left[
\begin{array}{cc|c}
1 & 0 & \alpha\\
0 & 0 & 1
\end{array}
\right]
\qquad\text{o}\qquad
\left[
\begin{array}{cc|c}
0 & 1 & \beta\\
0 & 0 & 1
\end{array}
\right]
\]

\noindent\textbf{(C) Infinitas soluciones (recta), rango $=1$:}
\[
\left[
\begin{array}{cc|c}
1 & \lambda & \mu\\
0 & 0 & 0
\end{array}
\right]
\qquad\text{o}\qquad
\left[
\begin{array}{cc|c}
0 & 1 & \mu\\
0 & 0 & 0
\end{array}
\right]
\]

% ---------------- 3x3 ----------------
\subsection*{3$\times$3 \; (tres planos en $\mathbb{R}^3$): \; $[A\mid b]\in\mathbb{R}^{3\times 4}$}

\noindent\textbf{(A) Solución única (punto), rango $=3$:}
\[
\left[
\begin{array}{ccc|c}
1 & 0 & 0 & \alpha\\
0 & 1 & 0 & \beta\\
0 & 0 & 1 & \gamma
\end{array}
\right]
\]
\begin{center}
    \adjincludegraphics[width=.85\linewidth,keepaspectratio]{../assets/planos/3×3 Se cortan en un punto (solución única).png}
\end{center}
\noindent\textbf{(B) Intersección en una recta, rango $=2$:}
\[
\left[
\begin{array}{ccc|c}
1 & 0 & a & b\\
0 & 1 & c & d\\
0 & 0 & 0 & 0
\end{array}
\right]
\]
\begin{center}
    \adjincludegraphics[width=.85\linewidth,keepaspectratio]{../assets/planos/3×3 Se cortan en una recta (consistente, rango 2).png}
\end{center}
\noindent\textbf{(C) Conjunto solución es un plano (coincidentes), rango $=1$:}
\[
\left[
\begin{array}{ccc|c}
1 & a & b & c\\
0 & 0 & 0 & 0\\
0 & 0 & 0 & 0
\end{array}
\right]
\]

\begin{center}
    \adjincludegraphics[width=.85\linewidth,keepaspectratio]{../assets/planos/3×3 Un plano de soluciones (rango 1; los tres coinciden).png}
\end{center}

\noindent\textbf{(D) Inconsistente (sin punto común):}
\[
\left[
\begin{array}{ccc|c}
1 & 0 & a & b\\
0 & 1 & c & d\\
0 & 0 & 0 & 1
\end{array}
\right]
\]
\begin{center}
    \adjincludegraphics[width=.85\linewidth,keepaspectratio]{../assets/planos/3×3 Dos paralelos y uno transversal.png}
\end{center}
% ---------------- 2x3 ----------------
\subsection*{2$\times$3 \; (dos planos en $\mathbb{R}^3$): \; $[A\mid b]\in\mathbb{R}^{2\times 4}$}

\noindent\textbf{(A) Intersección en una recta, rango $=2$:}
\[
\left[
\begin{array}{ccc|c}
1 & 0 & a & b\\
0 & 1 & c & d
\end{array}
\right]
\]

\begin{center}
    \adjincludegraphics[width=.85\linewidth,keepaspectratio]{../assets/planos/2x3 Se cortan en un punto (solución única).png}
\end{center}

\noindent\textbf{(B) Coincidentes (plano de soluciones), rango $=1$:}
\[
\left[
\begin{array}{ccc|c}
1 & a & b & c\\
0 & 0 & 0 & 0
\end{array}
\right]
\]

\noindent\textbf{(C) Paralelos distintos (inconsistente):}
\[
\left[
\begin{array}{ccc|c}
1 & a & b & c\\
0 & 0 & 0 & 1
\end{array}
\right]
\]

% ---------------- 3x2 ----------------
\subsection*{3$\times$2 \; (tres rectas en $\mathbb{R}^2$): \; $[A\mid b]\in\mathbb{R}^{3\times 3}$}

\noindent\textbf{(A) Consistente con solución única (tres concurrentes), rango $=2$:}
\[
\left[
\begin{array}{cc|c}
1 & 0 & \alpha\\
0 & 1 & \beta\\
0 & 0 & 0
\end{array}
\right]
\]
\begin{center}
    \adjincludegraphics[width=.85\linewidth,keepaspectratio]{../assets/planos/3×2 Consistente con solución única (tres concurrentes en un punto)}
\end{center}
\noindent\textbf{(B) Inconsistente (sin punto común):}
\[
\left[
\begin{array}{cc|c}
1 & 0 & \alpha\\
0 & 1 & \beta\\
0 & 0 & 1
\end{array}
\right]
\]

\noindent\textbf{(C) Infinitas soluciones (todas coinciden: una recta), rango $=1$:}
\[
\left[
\begin{array}{cc|c}
1 & \lambda & \mu\\
0 & 0 & 0\\
0 & 0 & 0
\end{array}
\right]
\]

\end{document}
